
% Default to the notebook output style

    


% Inherit from the specified cell style.




    
\documentclass[11pt]{article}

    
    
    \usepackage[T1]{fontenc}
    % Nicer default font (+ math font) than Computer Modern for most use cases
    \usepackage{mathpazo}

    % Basic figure setup, for now with no caption control since it's done
    % automatically by Pandoc (which extracts ![](path) syntax from Markdown).
    \usepackage{graphicx}
    % We will generate all images so they have a width \maxwidth. This means
    % that they will get their normal width if they fit onto the page, but
    % are scaled down if they would overflow the margins.
    \makeatletter
    \def\maxwidth{\ifdim\Gin@nat@width>\linewidth\linewidth
    \else\Gin@nat@width\fi}
    \makeatother
    \let\Oldincludegraphics\includegraphics
    % Set max figure width to be 80% of text width, for now hardcoded.
    \renewcommand{\includegraphics}[1]{\Oldincludegraphics[width=.8\maxwidth]{#1}}
    % Ensure that by default, figures have no caption (until we provide a
    % proper Figure object with a Caption API and a way to capture that
    % in the conversion process - todo).
    \usepackage{caption}
    \DeclareCaptionLabelFormat{nolabel}{}
    \captionsetup{labelformat=nolabel}

    \usepackage{adjustbox} % Used to constrain images to a maximum size 
    \usepackage{xcolor} % Allow colors to be defined
    \usepackage{enumerate} % Needed for markdown enumerations to work
    \usepackage{geometry} % Used to adjust the document margins
    \usepackage{amsmath} % Equations
    \usepackage{amssymb} % Equations
    \usepackage{textcomp} % defines textquotesingle
    % Hack from http://tex.stackexchange.com/a/47451/13684:
    \AtBeginDocument{%
        \def\PYZsq{\textquotesingle}% Upright quotes in Pygmentized code
    }
    \usepackage{upquote} % Upright quotes for verbatim code
    \usepackage{eurosym} % defines \euro
    \usepackage[mathletters]{ucs} % Extended unicode (utf-8) support
    \usepackage[utf8x]{inputenc} % Allow utf-8 characters in the tex document
    \usepackage{fancyvrb} % verbatim replacement that allows latex
    \usepackage{grffile} % extends the file name processing of package graphics 
                         % to support a larger range 
    % The hyperref package gives us a pdf with properly built
    % internal navigation ('pdf bookmarks' for the table of contents,
    % internal cross-reference links, web links for URLs, etc.)
    \usepackage{hyperref}
    \usepackage{longtable} % longtable support required by pandoc >1.10
    \usepackage{booktabs}  % table support for pandoc > 1.12.2
    \usepackage[inline]{enumitem} % IRkernel/repr support (it uses the enumerate* environment)
    \usepackage[normalem]{ulem} % ulem is needed to support strikethroughs (\sout)
                                % normalem makes italics be italics, not underlines
    

    
    
    % Colors for the hyperref package
    \definecolor{urlcolor}{rgb}{0,.145,.698}
    \definecolor{linkcolor}{rgb}{.71,0.21,0.01}
    \definecolor{citecolor}{rgb}{.12,.54,.11}

    % ANSI colors
    \definecolor{ansi-black}{HTML}{3E424D}
    \definecolor{ansi-black-intense}{HTML}{282C36}
    \definecolor{ansi-red}{HTML}{E75C58}
    \definecolor{ansi-red-intense}{HTML}{B22B31}
    \definecolor{ansi-green}{HTML}{00A250}
    \definecolor{ansi-green-intense}{HTML}{007427}
    \definecolor{ansi-yellow}{HTML}{DDB62B}
    \definecolor{ansi-yellow-intense}{HTML}{B27D12}
    \definecolor{ansi-blue}{HTML}{208FFB}
    \definecolor{ansi-blue-intense}{HTML}{0065CA}
    \definecolor{ansi-magenta}{HTML}{D160C4}
    \definecolor{ansi-magenta-intense}{HTML}{A03196}
    \definecolor{ansi-cyan}{HTML}{60C6C8}
    \definecolor{ansi-cyan-intense}{HTML}{258F8F}
    \definecolor{ansi-white}{HTML}{C5C1B4}
    \definecolor{ansi-white-intense}{HTML}{A1A6B2}

    % commands and environments needed by pandoc snippets
    % extracted from the output of `pandoc -s`
    \providecommand{\tightlist}{%
      \setlength{\itemsep}{0pt}\setlength{\parskip}{0pt}}
    \DefineVerbatimEnvironment{Highlighting}{Verbatim}{commandchars=\\\{\}}
    % Add ',fontsize=\small' for more characters per line
    \newenvironment{Shaded}{}{}
    \newcommand{\KeywordTok}[1]{\textcolor[rgb]{0.00,0.44,0.13}{\textbf{{#1}}}}
    \newcommand{\DataTypeTok}[1]{\textcolor[rgb]{0.56,0.13,0.00}{{#1}}}
    \newcommand{\DecValTok}[1]{\textcolor[rgb]{0.25,0.63,0.44}{{#1}}}
    \newcommand{\BaseNTok}[1]{\textcolor[rgb]{0.25,0.63,0.44}{{#1}}}
    \newcommand{\FloatTok}[1]{\textcolor[rgb]{0.25,0.63,0.44}{{#1}}}
    \newcommand{\CharTok}[1]{\textcolor[rgb]{0.25,0.44,0.63}{{#1}}}
    \newcommand{\StringTok}[1]{\textcolor[rgb]{0.25,0.44,0.63}{{#1}}}
    \newcommand{\CommentTok}[1]{\textcolor[rgb]{0.38,0.63,0.69}{\textit{{#1}}}}
    \newcommand{\OtherTok}[1]{\textcolor[rgb]{0.00,0.44,0.13}{{#1}}}
    \newcommand{\AlertTok}[1]{\textcolor[rgb]{1.00,0.00,0.00}{\textbf{{#1}}}}
    \newcommand{\FunctionTok}[1]{\textcolor[rgb]{0.02,0.16,0.49}{{#1}}}
    \newcommand{\RegionMarkerTok}[1]{{#1}}
    \newcommand{\ErrorTok}[1]{\textcolor[rgb]{1.00,0.00,0.00}{\textbf{{#1}}}}
    \newcommand{\NormalTok}[1]{{#1}}
    
    % Additional commands for more recent versions of Pandoc
    \newcommand{\ConstantTok}[1]{\textcolor[rgb]{0.53,0.00,0.00}{{#1}}}
    \newcommand{\SpecialCharTok}[1]{\textcolor[rgb]{0.25,0.44,0.63}{{#1}}}
    \newcommand{\VerbatimStringTok}[1]{\textcolor[rgb]{0.25,0.44,0.63}{{#1}}}
    \newcommand{\SpecialStringTok}[1]{\textcolor[rgb]{0.73,0.40,0.53}{{#1}}}
    \newcommand{\ImportTok}[1]{{#1}}
    \newcommand{\DocumentationTok}[1]{\textcolor[rgb]{0.73,0.13,0.13}{\textit{{#1}}}}
    \newcommand{\AnnotationTok}[1]{\textcolor[rgb]{0.38,0.63,0.69}{\textbf{\textit{{#1}}}}}
    \newcommand{\CommentVarTok}[1]{\textcolor[rgb]{0.38,0.63,0.69}{\textbf{\textit{{#1}}}}}
    \newcommand{\VariableTok}[1]{\textcolor[rgb]{0.10,0.09,0.49}{{#1}}}
    \newcommand{\ControlFlowTok}[1]{\textcolor[rgb]{0.00,0.44,0.13}{\textbf{{#1}}}}
    \newcommand{\OperatorTok}[1]{\textcolor[rgb]{0.40,0.40,0.40}{{#1}}}
    \newcommand{\BuiltInTok}[1]{{#1}}
    \newcommand{\ExtensionTok}[1]{{#1}}
    \newcommand{\PreprocessorTok}[1]{\textcolor[rgb]{0.74,0.48,0.00}{{#1}}}
    \newcommand{\AttributeTok}[1]{\textcolor[rgb]{0.49,0.56,0.16}{{#1}}}
    \newcommand{\InformationTok}[1]{\textcolor[rgb]{0.38,0.63,0.69}{\textbf{\textit{{#1}}}}}
    \newcommand{\WarningTok}[1]{\textcolor[rgb]{0.38,0.63,0.69}{\textbf{\textit{{#1}}}}}
    
    
    % Define a nice break command that doesn't care if a line doesn't already
    % exist.
    \def\br{\hspace*{\fill} \\* }
    % Math Jax compatability definitions
    \def\gt{>}
    \def\lt{<}
    % Document parameters
    \title{set\_3}
    
    
    

    % Pygments definitions
    
\makeatletter
\def\PY@reset{\let\PY@it=\relax \let\PY@bf=\relax%
    \let\PY@ul=\relax \let\PY@tc=\relax%
    \let\PY@bc=\relax \let\PY@ff=\relax}
\def\PY@tok#1{\csname PY@tok@#1\endcsname}
\def\PY@toks#1+{\ifx\relax#1\empty\else%
    \PY@tok{#1}\expandafter\PY@toks\fi}
\def\PY@do#1{\PY@bc{\PY@tc{\PY@ul{%
    \PY@it{\PY@bf{\PY@ff{#1}}}}}}}
\def\PY#1#2{\PY@reset\PY@toks#1+\relax+\PY@do{#2}}

\expandafter\def\csname PY@tok@gd\endcsname{\def\PY@tc##1{\textcolor[rgb]{0.63,0.00,0.00}{##1}}}
\expandafter\def\csname PY@tok@gu\endcsname{\let\PY@bf=\textbf\def\PY@tc##1{\textcolor[rgb]{0.50,0.00,0.50}{##1}}}
\expandafter\def\csname PY@tok@gt\endcsname{\def\PY@tc##1{\textcolor[rgb]{0.00,0.27,0.87}{##1}}}
\expandafter\def\csname PY@tok@gs\endcsname{\let\PY@bf=\textbf}
\expandafter\def\csname PY@tok@gr\endcsname{\def\PY@tc##1{\textcolor[rgb]{1.00,0.00,0.00}{##1}}}
\expandafter\def\csname PY@tok@cm\endcsname{\let\PY@it=\textit\def\PY@tc##1{\textcolor[rgb]{0.25,0.50,0.50}{##1}}}
\expandafter\def\csname PY@tok@vg\endcsname{\def\PY@tc##1{\textcolor[rgb]{0.10,0.09,0.49}{##1}}}
\expandafter\def\csname PY@tok@vi\endcsname{\def\PY@tc##1{\textcolor[rgb]{0.10,0.09,0.49}{##1}}}
\expandafter\def\csname PY@tok@vm\endcsname{\def\PY@tc##1{\textcolor[rgb]{0.10,0.09,0.49}{##1}}}
\expandafter\def\csname PY@tok@mh\endcsname{\def\PY@tc##1{\textcolor[rgb]{0.40,0.40,0.40}{##1}}}
\expandafter\def\csname PY@tok@cs\endcsname{\let\PY@it=\textit\def\PY@tc##1{\textcolor[rgb]{0.25,0.50,0.50}{##1}}}
\expandafter\def\csname PY@tok@ge\endcsname{\let\PY@it=\textit}
\expandafter\def\csname PY@tok@vc\endcsname{\def\PY@tc##1{\textcolor[rgb]{0.10,0.09,0.49}{##1}}}
\expandafter\def\csname PY@tok@il\endcsname{\def\PY@tc##1{\textcolor[rgb]{0.40,0.40,0.40}{##1}}}
\expandafter\def\csname PY@tok@go\endcsname{\def\PY@tc##1{\textcolor[rgb]{0.53,0.53,0.53}{##1}}}
\expandafter\def\csname PY@tok@cp\endcsname{\def\PY@tc##1{\textcolor[rgb]{0.74,0.48,0.00}{##1}}}
\expandafter\def\csname PY@tok@gi\endcsname{\def\PY@tc##1{\textcolor[rgb]{0.00,0.63,0.00}{##1}}}
\expandafter\def\csname PY@tok@gh\endcsname{\let\PY@bf=\textbf\def\PY@tc##1{\textcolor[rgb]{0.00,0.00,0.50}{##1}}}
\expandafter\def\csname PY@tok@ni\endcsname{\let\PY@bf=\textbf\def\PY@tc##1{\textcolor[rgb]{0.60,0.60,0.60}{##1}}}
\expandafter\def\csname PY@tok@nl\endcsname{\def\PY@tc##1{\textcolor[rgb]{0.63,0.63,0.00}{##1}}}
\expandafter\def\csname PY@tok@nn\endcsname{\let\PY@bf=\textbf\def\PY@tc##1{\textcolor[rgb]{0.00,0.00,1.00}{##1}}}
\expandafter\def\csname PY@tok@no\endcsname{\def\PY@tc##1{\textcolor[rgb]{0.53,0.00,0.00}{##1}}}
\expandafter\def\csname PY@tok@na\endcsname{\def\PY@tc##1{\textcolor[rgb]{0.49,0.56,0.16}{##1}}}
\expandafter\def\csname PY@tok@nb\endcsname{\def\PY@tc##1{\textcolor[rgb]{0.00,0.50,0.00}{##1}}}
\expandafter\def\csname PY@tok@nc\endcsname{\let\PY@bf=\textbf\def\PY@tc##1{\textcolor[rgb]{0.00,0.00,1.00}{##1}}}
\expandafter\def\csname PY@tok@nd\endcsname{\def\PY@tc##1{\textcolor[rgb]{0.67,0.13,1.00}{##1}}}
\expandafter\def\csname PY@tok@ne\endcsname{\let\PY@bf=\textbf\def\PY@tc##1{\textcolor[rgb]{0.82,0.25,0.23}{##1}}}
\expandafter\def\csname PY@tok@nf\endcsname{\def\PY@tc##1{\textcolor[rgb]{0.00,0.00,1.00}{##1}}}
\expandafter\def\csname PY@tok@si\endcsname{\let\PY@bf=\textbf\def\PY@tc##1{\textcolor[rgb]{0.73,0.40,0.53}{##1}}}
\expandafter\def\csname PY@tok@s2\endcsname{\def\PY@tc##1{\textcolor[rgb]{0.73,0.13,0.13}{##1}}}
\expandafter\def\csname PY@tok@nt\endcsname{\let\PY@bf=\textbf\def\PY@tc##1{\textcolor[rgb]{0.00,0.50,0.00}{##1}}}
\expandafter\def\csname PY@tok@nv\endcsname{\def\PY@tc##1{\textcolor[rgb]{0.10,0.09,0.49}{##1}}}
\expandafter\def\csname PY@tok@s1\endcsname{\def\PY@tc##1{\textcolor[rgb]{0.73,0.13,0.13}{##1}}}
\expandafter\def\csname PY@tok@dl\endcsname{\def\PY@tc##1{\textcolor[rgb]{0.73,0.13,0.13}{##1}}}
\expandafter\def\csname PY@tok@ch\endcsname{\let\PY@it=\textit\def\PY@tc##1{\textcolor[rgb]{0.25,0.50,0.50}{##1}}}
\expandafter\def\csname PY@tok@m\endcsname{\def\PY@tc##1{\textcolor[rgb]{0.40,0.40,0.40}{##1}}}
\expandafter\def\csname PY@tok@gp\endcsname{\let\PY@bf=\textbf\def\PY@tc##1{\textcolor[rgb]{0.00,0.00,0.50}{##1}}}
\expandafter\def\csname PY@tok@sh\endcsname{\def\PY@tc##1{\textcolor[rgb]{0.73,0.13,0.13}{##1}}}
\expandafter\def\csname PY@tok@ow\endcsname{\let\PY@bf=\textbf\def\PY@tc##1{\textcolor[rgb]{0.67,0.13,1.00}{##1}}}
\expandafter\def\csname PY@tok@sx\endcsname{\def\PY@tc##1{\textcolor[rgb]{0.00,0.50,0.00}{##1}}}
\expandafter\def\csname PY@tok@bp\endcsname{\def\PY@tc##1{\textcolor[rgb]{0.00,0.50,0.00}{##1}}}
\expandafter\def\csname PY@tok@c1\endcsname{\let\PY@it=\textit\def\PY@tc##1{\textcolor[rgb]{0.25,0.50,0.50}{##1}}}
\expandafter\def\csname PY@tok@fm\endcsname{\def\PY@tc##1{\textcolor[rgb]{0.00,0.00,1.00}{##1}}}
\expandafter\def\csname PY@tok@o\endcsname{\def\PY@tc##1{\textcolor[rgb]{0.40,0.40,0.40}{##1}}}
\expandafter\def\csname PY@tok@kc\endcsname{\let\PY@bf=\textbf\def\PY@tc##1{\textcolor[rgb]{0.00,0.50,0.00}{##1}}}
\expandafter\def\csname PY@tok@c\endcsname{\let\PY@it=\textit\def\PY@tc##1{\textcolor[rgb]{0.25,0.50,0.50}{##1}}}
\expandafter\def\csname PY@tok@mf\endcsname{\def\PY@tc##1{\textcolor[rgb]{0.40,0.40,0.40}{##1}}}
\expandafter\def\csname PY@tok@err\endcsname{\def\PY@bc##1{\setlength{\fboxsep}{0pt}\fcolorbox[rgb]{1.00,0.00,0.00}{1,1,1}{\strut ##1}}}
\expandafter\def\csname PY@tok@mb\endcsname{\def\PY@tc##1{\textcolor[rgb]{0.40,0.40,0.40}{##1}}}
\expandafter\def\csname PY@tok@ss\endcsname{\def\PY@tc##1{\textcolor[rgb]{0.10,0.09,0.49}{##1}}}
\expandafter\def\csname PY@tok@sr\endcsname{\def\PY@tc##1{\textcolor[rgb]{0.73,0.40,0.53}{##1}}}
\expandafter\def\csname PY@tok@mo\endcsname{\def\PY@tc##1{\textcolor[rgb]{0.40,0.40,0.40}{##1}}}
\expandafter\def\csname PY@tok@kd\endcsname{\let\PY@bf=\textbf\def\PY@tc##1{\textcolor[rgb]{0.00,0.50,0.00}{##1}}}
\expandafter\def\csname PY@tok@mi\endcsname{\def\PY@tc##1{\textcolor[rgb]{0.40,0.40,0.40}{##1}}}
\expandafter\def\csname PY@tok@kn\endcsname{\let\PY@bf=\textbf\def\PY@tc##1{\textcolor[rgb]{0.00,0.50,0.00}{##1}}}
\expandafter\def\csname PY@tok@cpf\endcsname{\let\PY@it=\textit\def\PY@tc##1{\textcolor[rgb]{0.25,0.50,0.50}{##1}}}
\expandafter\def\csname PY@tok@kr\endcsname{\let\PY@bf=\textbf\def\PY@tc##1{\textcolor[rgb]{0.00,0.50,0.00}{##1}}}
\expandafter\def\csname PY@tok@s\endcsname{\def\PY@tc##1{\textcolor[rgb]{0.73,0.13,0.13}{##1}}}
\expandafter\def\csname PY@tok@kp\endcsname{\def\PY@tc##1{\textcolor[rgb]{0.00,0.50,0.00}{##1}}}
\expandafter\def\csname PY@tok@w\endcsname{\def\PY@tc##1{\textcolor[rgb]{0.73,0.73,0.73}{##1}}}
\expandafter\def\csname PY@tok@kt\endcsname{\def\PY@tc##1{\textcolor[rgb]{0.69,0.00,0.25}{##1}}}
\expandafter\def\csname PY@tok@sc\endcsname{\def\PY@tc##1{\textcolor[rgb]{0.73,0.13,0.13}{##1}}}
\expandafter\def\csname PY@tok@sb\endcsname{\def\PY@tc##1{\textcolor[rgb]{0.73,0.13,0.13}{##1}}}
\expandafter\def\csname PY@tok@sa\endcsname{\def\PY@tc##1{\textcolor[rgb]{0.73,0.13,0.13}{##1}}}
\expandafter\def\csname PY@tok@k\endcsname{\let\PY@bf=\textbf\def\PY@tc##1{\textcolor[rgb]{0.00,0.50,0.00}{##1}}}
\expandafter\def\csname PY@tok@se\endcsname{\let\PY@bf=\textbf\def\PY@tc##1{\textcolor[rgb]{0.73,0.40,0.13}{##1}}}
\expandafter\def\csname PY@tok@sd\endcsname{\let\PY@it=\textit\def\PY@tc##1{\textcolor[rgb]{0.73,0.13,0.13}{##1}}}

\def\PYZbs{\char`\\}
\def\PYZus{\char`\_}
\def\PYZob{\char`\{}
\def\PYZcb{\char`\}}
\def\PYZca{\char`\^}
\def\PYZam{\char`\&}
\def\PYZlt{\char`\<}
\def\PYZgt{\char`\>}
\def\PYZsh{\char`\#}
\def\PYZpc{\char`\%}
\def\PYZdl{\char`\$}
\def\PYZhy{\char`\-}
\def\PYZsq{\char`\'}
\def\PYZdq{\char`\"}
\def\PYZti{\char`\~}
% for compatibility with earlier versions
\def\PYZat{@}
\def\PYZlb{[}
\def\PYZrb{]}
\makeatother


    % Exact colors from NB
    \definecolor{incolor}{rgb}{0.0, 0.0, 0.5}
    \definecolor{outcolor}{rgb}{0.545, 0.0, 0.0}



    
    % Prevent overflowing lines due to hard-to-break entities
    \sloppy 
    % Setup hyperref package
    \hypersetup{
      breaklinks=true,  % so long urls are correctly broken across lines
      colorlinks=true,
      urlcolor=urlcolor,
      linkcolor=linkcolor,
      citecolor=citecolor,
      }
    % Slightly bigger margins than the latex defaults
    
    \geometry{verbose,tmargin=1in,bmargin=1in,lmargin=1in,rmargin=1in}
    
    

    \begin{document}
    
    
    \maketitle
    
    

    
    \subsection{6.832: Problem Set \#3}\label{problem-set-3}

Due on Friday, March 16, 2018 at 17:00. See course website for
submission details. Use Drake release tag \texttt{drake-20180307}, i.e.
use this notebook via
\texttt{./docker\_run\_notebook.sh\ drake-20180307\ .}, or whichever
script you need for your platform.

To submit for autograding, upload this file, \emph{and also the
inertial\_wheel\_pendulum.py and
inertial\_wheel\_pendulum\_visualizer.py files supplied to you with any
modifications you have made}, to the "Problem Set 3, Code Submission"
assignment.

\begin{center}\rule{0.5\linewidth}{\linethickness}\end{center}

    \subsection{About this problem set}\label{about-this-problem-set}

This problem set will entirely live inside this jupyter notebook.

Grades will be assigned based on three components:

\begin{itemize}
\tightlist
\item
  \textbf{Manually graded free-response questions} -\/- the TAs will
  manually assign grades to your answers to short answer responses. You
  can write inline responses using
  \href{https://github.com/adam-p/markdown-here/wiki/Markdown-Cheatsheet}{Markdown}
  with inline LaTeX -\/- double-click on any problem writeup to see some
  examples. Double-click response areas to edit them, and press
  Control-Enter to finish editing them.
\item
  \textbf{Automated code testing} -\/- we will run automated tests
  against specific functions (see more details when we introduce the
  first coding test).
\item
  \textbf{Quick code review} -\/- we will perform a quick manual check
  to make sure you have actually implemented the functions correctly (as
  opposed to hacked the unit tests to pass!).
\end{itemize}

\begin{center}\rule{0.5\linewidth}{\linethickness}\end{center}

    \section{1. Lyapunov Theory}\label{lyapunov-theory}

The following are a series of short answer problems meant to test your
understanding of Lyapunov functions and Lyapunov theory.

    \begin{center}\rule{0.5\linewidth}{\linethickness}\end{center}

\subsubsection{Quick check for
understanding}\label{quick-check-for-understanding}

\paragraph{We recommend you stop for a second to make sure you can
answer the following
questions:}\label{we-recommend-you-stop-for-a-second-to-make-sure-you-can-answer-the-following-questions}

\begin{enumerate}
\def\labelenumi{\arabic{enumi}.}
\item
  What is a Lyapunov function?
\item
  How does Lyanpunov theory relate to robots?
\item
  What is the relationship between Lyapunov functions -\/- we have been
  writing them as \(V(x)\) -\/- and cost-to-go functions -\/- we have
  been writing them as \(J(x)\)?
\item
  What are the conditions for making a Lyapunov function valid?
\item
  Just a function \(V(x)\) doesn't itself demand anything about how the
  function will evolve over time. How do we bring the idea of evolution
  over time into our Lyapunov analysis?
\item
  If somebody gives you a Lyapunov function \(V(x)\) and a dynamical
  system \(\dot{x} = f(x)\), is it hard to check if \(V\) is a valid
  Lyapunov function?
\item
  If you would like to find a Lyapunov function on your own, how might
  you do that?
\item
  Can you prove the nonexistence of a Lyapunov function, for a
  particular system?
\end{enumerate}

(We won't grade answers to these questions, but are happy to talk about
your answers. Office hours are the best opportunities to discuss.)

\begin{center}\rule{0.5\linewidth}{\linethickness}\end{center}

    \subsection{1.1 (2 points)}\label{points}

A dynamical system is defined by the equations
\(\dot{x_1}=f_1(x_1,x_2)\) and \(\dot{x_2}=f_2(x_1,x_2)\). You find that
the function \(V=\frac{1}{2}x_1^2\) has a derivative
\(\dot{V} = \left[\dfrac{dV}{dx_1}, \dfrac{dV}{dx_2}\right] \left[ f_1, f_2 \right]^T\)
which is negative semidefinite. What, if anything, can you prove about
the behavior of this system with this function \(V\)?

    \textbf{Short answer explanation for 1.1.}

It is stable in the sense of Lyapunov.

    \subsection{1.2 (4 points)}\label{points}

For the system

\begin{align}
\dot{x_1}&=-\frac{6x_1}{(1+x_1^2)^2}+2x_2  \\
\dot{x_2}&=-\frac{2(x_1+x_2)}{(1+x_1^2)^2}
\end{align}

you are given the positive definite function
\(V(x) =\frac{x_1^2}{1 + x_1^2}+ x_2^2\) and told that, for this system,
\(\dot{V}\) is negative definite over the entire space. Is \(V\) a valid
Lyapunov function which proves global asymptotic stability to the origin
for the system described by these equations? Why or why not? Hint:
Trying simulating a few trajectories of this system or plotting the
vector field to build more intuition before answering this problem.

    \begin{Verbatim}[commandchars=\\\{\}]
{\color{incolor}In [{\color{incolor} }]:} \PY{c+c1}{\PYZsh{} Sandbox for doing math, plotting}
        \PY{c+c1}{\PYZsh{} You might find this reference useful for making quiver plots / plotting vector fields:}
        \PY{c+c1}{\PYZsh{} https://matplotlib.org/examples/pylab\PYZus{}examples/quiver\PYZus{}demo.html}
\end{Verbatim}


    \textbf{Short answer explanation for 1.2.}

\[\dot{V} = \frac{2 x_1}/{(1 + x_1^2)^2}\left[-\frac{6x_1}{(1+x_1^2)^2}+2x_2 \right]   - 2 x_2 \frac{2(x_1+x_2)}{(1+x_1^2)^2}\]

\[\dot{V} = \frac{-2}{(1 + x_1^2)^2} \left[ \frac{6x_1^2}{(1+x_1^2)^2}+2x_1 x_2 + 2 x_1 x_2 + 2x_2^2 \right]\]

YOUR ANSWER HERE

    \subsection{1.3 (2 points)}\label{points}

Given dynamics \(\dot{x} = f(x)\) and a Lyapunov candidate \(V(x)\), let
\(D\) be any domain such that for all \(\hat{x} \in D\), \(V(\hat{x})\)
is positive definite and \(\dot{V}(\hat{x})\) is negative definite.

\begin{enumerate}
\def\labelenumi{\arabic{enumi})}
\item
  Can we say that all initial conditions inside \(D\) will stay inside
  \(D\)? Why or why not?
\item
  Can we say that all initial conditions inside \(D\) will converge to
  the origin? Why or why not?
\end{enumerate}

    \textbf{Short answer explanation for 1.3.}

\begin{enumerate}
\def\labelenumi{\arabic{enumi})}
\item
  No, suppose D contains only a single point \(x_1\). You can easily
  construct \(f\), \(V\) such that \(x\) will leave the region (namely a
  nonzero function \(f\)).
\item
  No, you can construct a point which asymptotes to a non-zero value
  (i.e. consider a negative expoential with a nonzero offset).
\end{enumerate}

    \subsection{1.4 (2 points)}\label{points}

If \(V_1(x)\) and \(V_2(x)\) are valid Lyapunov functions that prove
global stability of a system to the origin, does \(V_1(x)\) necessarily
equal \(V_2(x)\)? In other words, are Lyapunov functions unique? Why or
why not?

    \textbf{Short answer explanation for 1.4.}

No, consider multiplying any V by a positive constant to get a new
function.

    \subsection{1.5 (4 points)}\label{points}

Consider the system given by

\begin{equation}
\dot{x}= \left(\begin{array}{c} {x_2} - {{x_1}}^3\\  - {{x_2}}^3 - {x_1} \end{array}\right)\label{p2System}
\end{equation}

Show that the Lyapunov function

\[
V(x) = x_1^2 + x_2^2 
\]

proves global asymptotic stability of the above equation to the origin.

    \textbf{Short answer explanation for 1.5.}

\begin{align}
V' &= \nabla V \cdot \dot{x}\\
&= -2(x_1^4+x_2^4)
\end{align}

Clearly from this \(V'\) is negative definite everywhere (except zero at
origin), and V is positive definite (except zero at the origin). Thus it
is globally stable as \(V \rightarrow \infty\) as
\(x \rightarrow \infty\).

    \section{2. Sums of Squares (SOS) for Lyapunov
functions}\label{sums-of-squares-sos-for-lyapunov-functions}

    \begin{center}\rule{0.5\linewidth}{\linethickness}\end{center}

\subsubsection{Quick check for
understanding}\label{quick-check-for-understanding}

\paragraph{We recommend you stop for a second to make sure you can
answer the following
questions:}\label{we-recommend-you-stop-for-a-second-to-make-sure-you-can-answer-the-following-questions}

\begin{enumerate}
\def\labelenumi{\arabic{enumi}.}
\item
  What is the simplest example of a function that is a sum of squares?
\item
  What is arguably the most useful property of functions that are sums
  of squares? When are they positive? When are they negative?
\item
  Are there functions that also have this property, but do not belong to
  the set of sums of squares? (Hint: Google the Motzkin polynomial
  mentioned in class.)
\item
  How do sums of squares (SOS) relate to Lyapunov functions?
\item
  Specifically, which property (or properties) of Lyapunov functions do
  we relate to them being sums of squares or not?
\item
  To make an expressive set of functions over our state space, we choose
  some set of monomials \(m(x)\). What are some examples of good choices
  for \(m(x)\)?
\item
  What does the form \(V(x) = m^T(x) Q m^T(x)\) have to do with a
  function being sums of squares?
\item
  Searching for \(Q\) in the above question can be formulated as a
  Semidefinite Program. What is the difference between Quadratic
  Programming and Semidefinite Programming?
\item
  If we find, using SOS optimization, a valid Lyapunov function, then
  what does this say about our system?
\item
  If we cannot find, using SOS optimization, a valid Lyapunov function,
  then what does this say about our system?
\item
  We are using SOS to help find Lyapunov functions. How does this help
  our robots?
\end{enumerate}

(We won't grade answers to these questions, but are happy to talk about
your answers. Office hours are the best opportunities to discuss.)

\begin{center}\rule{0.5\linewidth}{\linethickness}\end{center}

    \subsection{2.1 (4 points, 4/4 autograded)}\label{points-44-autograded}

Consider the polynomial: \[
p(x_1,x_2) = 2x_1^4 + 2x_1^3x_2 - x_1^2x_2^2 + 5x_2^4.
\]

\textbf{Prove that this polynomial is nonnegative} by finding a
representation as follows:

\[
p(x_1,x_2) = \left[ \begin{array}{c}
x_1^2 \\
x_2^2 \\
x_1 x_2 \end{array} \right]^T
\ \ \
Q
\ \ \
\left[ \begin{array}{c}
x_1^2 \\
x_2^2 \\
x_1 x_2 \end{array} \right]
\]

where (Q) is given by: \[
Q = \left[ \begin{array}{ccc}
2, \ a, \ 1 \\
a, \ 5, \ 0 \\
1, \ 0, \ b \end{array} \right].
\]

Here (Q) must be positive semidefinite. \textbf{Type your value for
\(Q\) below.} Make sure that the resulting (Q) is positive semidefinite.

    \begin{Verbatim}[commandchars=\\\{\}]
{\color{incolor}In [{\color{incolor}2}]:} \PY{l+s+sd}{\PYZsq{}\PYZsq{}\PYZsq{}}
        \PY{l+s+sd}{User code for problem 2.1.}
        \PY{l+s+sd}{\PYZsq{}\PYZsq{}\PYZsq{}}
        \PY{k+kn}{import} \PY{n+nn}{numpy} \PY{k+kn}{as} \PY{n+nn}{np}
        \PY{k}{def} \PY{n+nf}{problem\PYZus{}2\PYZus{}1\PYZus{}get\PYZus{}Q}\PY{p}{(}\PY{p}{)}\PY{p}{:}
            \PY{c+c1}{\PYZsh{} Set a and b appropriately done by matching coefficients, and ensuring that the diagonal vector of Q \PYZgt{} 0}
            \PY{n}{a} \PY{o}{=} \PY{o}{\PYZhy{}}\PY{l+m+mi}{1}
            \PY{n}{b} \PY{o}{=} \PY{l+m+mi}{1}
            \PY{n}{Q} \PY{o}{=} \PY{n}{np}\PY{o}{.}\PY{n}{array}\PY{p}{(}\PY{p}{[}
                \PY{p}{[}\PY{l+m+mi}{2}\PY{p}{,} \PY{n}{a}\PY{p}{,} \PY{l+m+mi}{1}\PY{p}{]}\PY{p}{,}
                \PY{p}{[}\PY{n}{a}\PY{p}{,} \PY{l+m+mi}{5}\PY{p}{,} \PY{l+m+mi}{0}\PY{p}{]}\PY{p}{,}
                \PY{p}{[}\PY{l+m+mi}{1}\PY{p}{,} \PY{l+m+mi}{0}\PY{p}{,} \PY{n}{b}\PY{p}{]}
            \PY{p}{]}\PY{p}{)}
            \PY{k}{return} \PY{n}{Q}
        \PY{k}{print} \PY{l+s+s2}{\PYZdq{}}\PY{l+s+s2}{Q = }\PY{l+s+s2}{\PYZdq{}}\PY{p}{,} \PY{n}{problem\PYZus{}2\PYZus{}1\PYZus{}get\PYZus{}Q}\PY{p}{(}\PY{p}{)}
\end{Verbatim}


    \begin{Verbatim}[commandchars=\\\{\}]
Q =  [[ 2 -1  1]
 [-1  5  0]
 [ 1  0  1]]

    \end{Verbatim}

    Finding such positive semidefinite Q, for carefully (but usually
automatically) selected basis vectors, is at the heart of using SOS
techniques for verification.

The textbook demonstrates
\href{https://github.com/RussTedrake/underactuated/blob/master/src/lyapunov/cubic_polynomial.py}{verification
of the ROA of a cubic polynomial system} as an example (the example
system appearing in the latter half of the
\href{http://underactuated.csail.mit.edu/underactuated.html?chapter=lyapunov}{Lyapunov
chapter of the text}. This code can be copy-pasted into this notebook,
if you want to give it a try. A simpler example of framing the search
for a common Lyapunov function search for multiple linear systems is
\href{https://github.com/RussTedrake/underactuated/blob/master/src/lyapunov/linear_systems_common_lyapunov.py}{also
available} as a demonstration of semidefinite programming.

    \section{3. Inertial Wheel Pendulum}\label{inertial-wheel-pendulum}

In this question we'll use a classic underactuated system, the inertial
wheel pendulum, to do a case study in bringing together all of these
components:

\begin{itemize}
\tightlist
\item
  Stabilization at the upright position using LQR (and how to linearize
  around a fixed point)
\item
  Finding where is our LQR controller "good", i.e. computing the region
  of attraction (RoA)
\item
  Swinging up with a simple energy-shaping controller and switching to
  LQR when inside the RoA
\end{itemize}

An inertial wheel pendulum is a single-link pendulum with a
torque-controlled reaction wheel mounted at its end:

\emph{(Image source Ramirez-Neria, Mario, et al. "On the linear Active
Disturbance Rejection Control of the inertia wheel pendulum.")}

The angle of the pendulum is \(\theta_1\), and the angle of the reaction
wheel is \(\theta_2\). The only control input is a (bounded) torque
applied to the wheel, \(\tau\) \[
{x} = \left[ \begin{array}{c} q
\\ \dot q 
\end{array} \right]
\ \ \ \ \
{q} = \left[ \begin{array}{c}
\theta_1 \\
\theta_2 \end{array} \right]
\ \ \ \ \
{u} = \left[ \tau \right]
\ \ \ \ \
|\tau| \leq \tau_{max}
\]

Highly recommended to watch some videos of inertial wheel pendulums in
action. (Click on the pictures to see videos on YouTube.)

Single axis inertial wheel pendulum (similar to the model we will use)

Small desktop single axis inertial wheel pendulum

Dual axis inertial wheel pendulum

Triple axis inertial wheel pendulum (unlike the dual axis, this can also
stabilize yaw) 

    \begin{center}\rule{0.5\linewidth}{\linethickness}\end{center}

Filling in the manipulator equations for our single-link pendulum,

\[
M \ddot{q} + C(q, \dot{q})\dot{q} = \tau_g(q) + Bu \\
M = \left[ 
\begin{array}{c c} 
  m_1 l_1^2 + m_2 l_2^2 + I_1 + I_2 & I_2 \\
  I_2 & I_2
\end{array}
\right]
\ \ \ \ \ 
C = \left[
\begin{array}{c c}
 0 & 0 \\
 0 & 0
\end{array}
\right]\\
\tau_g(q) = \left[
\begin{array}{c}
 -(m_1 l_1 + m_2 l_2) g sin(\theta_1) \\ 0
\end{array}
\right]
\ \ \ \ \ 
B = \left[
\begin{array}{c}
 0 \\ 1
\end{array}
\right]
\]

\(\tau_g(q)\) should look familiar from the simple pendulum case. The
effect of \(M\) is little trickier, as it includes some interaction
terms between \(\theta_1\) and \(\theta_2\).
\href{http://home.deib.polimi.it/gini/robot/docs/spong.pdf}{This
textbook's} treatment of the double pendulum (which has fundamentally
similar inertia) in Chapter 9 might be of use if you want to dig into
this more thoroughly.

    \begin{center}\rule{0.5\linewidth}{\linethickness}\end{center}

\subsection{3.1 Linearization (4 points, 4/4
autograded)}\label{linearization-4-points-44-autograded}

We're interested in stabilizing this system to its upright fixed point
at \(\theta_1 = \pi\). Let's try using LQR to do this. As a first step,
calculate the linearization of these dynamics -\/- that is, find
functions \(A(x_f)\) and \(B(x_f)\) such that the linear system
\(\dot x = A(x_f)*\bar{x} + B(x_f)*\bar{u}\), for \(\bar{x} = x-x_f\)
and \(\bar{u} = u-u_f\), approximates our full system in the
neighborhood of the fixed point \(x_f, u_f\). \textbf{Derive those here,
and also fill out the corresponding function \(GetLinearizedDynamics\)
of \emph{inertial\_wheel\_pendulum.py} to generate these.}

    \begin{align}
      {\bf A}_{lin} =& \begin{bmatrix} {0} & {I} \\ M^{-1} \frac{\partial \tau_g}{\partial q} + \sum_{j} M^{-1}\frac{\partial B_j}{\partial q} u_j  & {0} \end{bmatrix}_{x=x^*,u=u^*} \\
      B_{lin} =& \begin{bmatrix} {0} \\ M^{-1} B
        \end{bmatrix}_{x=x^*, u=u^*}
      \end{align}

\begin{align}
{\bf A}_{lin} =&
\begin{bmatrix} {0} & {I} \\ (M^{-1})_{.1} * -(m_1 l_1 + m_2 l_2) g \cos(\theta_1)& {0} \end{bmatrix}_{x=x^*,u=u^*} \\
B_{lin} =& \begin{bmatrix}
{0} \\
M^{-1} B
\end{bmatrix}_{x=x^*, u=u^*}
\end{align}

    \begin{center}\rule{0.5\linewidth}{\linethickness}\end{center}

\subsubsection{Quick check for
understanding}\label{quick-check-for-understanding}

\paragraph{We recommend you stop for a second to make sure you can
answer the following
questions:}\label{we-recommend-you-stop-for-a-second-to-make-sure-you-can-answer-the-following-questions}

\begin{enumerate}
\def\labelenumi{\arabic{enumi}.}
\item
  What makes a system linear vs. nonlinear?
\item
  Are there edge cases where you are not sure if a system is nonlinear
  or linear?
\item
  What does \(\dot{x} = f(x,u)\) usually mean? What does
  \(\dot{x} = Ax + Bu\) usually mean?
\item
  Why would we linearize a system?
\item
  How would we determine when/where the linearization of the system is
  "good"?
\item
  What are eigenvectors and how do they relate to robots?
\end{enumerate}

(We won't grade answers to these questions, but are happy to talk about
your answers. Office hours are the best opportunities to discuss.)

\begin{center}\rule{0.5\linewidth}{\linethickness}\end{center}

    Note: you can use any text editor you like to edit
\texttt{inertial\_wheel\_pendulum.py} (vim, emacs, Sublime, Jupyter
notebook's built-in editor, etc). It is just a file inside the
\texttt{set\_3} folder. We are mounting it externally into the docker
container. Just make sure you save \texttt{inertial\_wheel\_pendulum.py}
before running the code cells below.

    \begin{Verbatim}[commandchars=\\\{\}]
{\color{incolor}In [{\color{incolor}58}]:} \PY{c+c1}{\PYZsh{} These IPython\PYZhy{}specific commands}
         \PY{c+c1}{\PYZsh{} tell the notebook to reload imported}
         \PY{c+c1}{\PYZsh{} files every time you rerun code. So}
         \PY{c+c1}{\PYZsh{} you can modify inertial\PYZus{}wheel\PYZus{}pendulum.py}
         \PY{c+c1}{\PYZsh{} and then rerun this cell to see the changes.}
         \PY{o}{\PYZpc{}}\PY{k}{load\PYZus{}ext} autoreload
         \PY{o}{\PYZpc{}}\PY{k}{autoreload} 2
         
         \PY{k+kn}{from} \PY{n+nn}{inertial\PYZus{}wheel\PYZus{}pendulum} \PY{k+kn}{import} \PY{o}{*}
         \PY{k+kn}{import} \PY{n+nn}{math}
         \PY{k+kn}{import} \PY{n+nn}{numpy} \PY{k+kn}{as} \PY{n+nn}{np}
         \PY{c+c1}{\PYZsh{} Make numpy printing prettier}
         \PY{n}{np}\PY{o}{.}\PY{n}{set\PYZus{}printoptions}\PY{p}{(}\PY{n}{precision}\PY{o}{=}\PY{l+m+mi}{3}\PY{p}{,} \PY{n}{suppress}\PY{o}{=}\PY{n+nb+bp}{True}\PY{p}{)}
         
         \PY{c+c1}{\PYZsh{} Define the upright fixed point here.}
         \PY{n}{uf} \PY{o}{=} \PY{n}{np}\PY{o}{.}\PY{n}{array}\PY{p}{(}\PY{p}{[}\PY{l+m+mf}{0.}\PY{p}{]}\PY{p}{)}
         \PY{n}{xf} \PY{o}{=} \PY{n}{np}\PY{o}{.}\PY{n}{array}\PY{p}{(}\PY{p}{[}\PY{n}{math}\PY{o}{.}\PY{n}{pi}\PY{p}{,} \PY{l+m+mi}{0}\PY{p}{,} \PY{l+m+mi}{0}\PY{p}{,} \PY{l+m+mi}{0}\PY{p}{]}\PY{p}{)}
         
         \PY{c+c1}{\PYZsh{} Pendulum params. You\PYZsq{}re free to play with these,}
         \PY{c+c1}{\PYZsh{} of course, but we\PYZsq{}ll be expecting you to use the}
         \PY{c+c1}{\PYZsh{} default values when answering questions, where}
         \PY{c+c1}{\PYZsh{} varying these values might make a difference.}
         \PY{n}{m1} \PY{o}{=} \PY{l+m+mf}{1.} \PY{c+c1}{\PYZsh{} Default 1}
         \PY{n}{l1} \PY{o}{=} \PY{l+m+mf}{1.} \PY{c+c1}{\PYZsh{} Default 1}
         \PY{n}{m2} \PY{o}{=} \PY{l+m+mf}{2.} \PY{c+c1}{\PYZsh{} Default 2}
         \PY{n}{l2} \PY{o}{=} \PY{l+m+mf}{2.} \PY{c+c1}{\PYZsh{} Default 2}
         \PY{n}{r} \PY{o}{=} \PY{l+m+mf}{1.0} \PY{c+c1}{\PYZsh{} Default 1}
         \PY{n}{g} \PY{o}{=} \PY{l+m+mi}{10}  \PY{c+c1}{\PYZsh{} Default 10}
         \PY{n}{input\PYZus{}max} \PY{o}{=} \PY{l+m+mi}{10}
         \PY{n}{pendulum\PYZus{}plant} \PY{o}{=} \PY{n}{InertialWheelPendulum}\PY{p}{(}
             \PY{n}{m1} \PY{o}{=} \PY{n}{m1}\PY{p}{,} \PY{n}{l1} \PY{o}{=} \PY{n}{l1}\PY{p}{,} \PY{n}{m2} \PY{o}{=} \PY{n}{m2}\PY{p}{,} \PY{n}{l2} \PY{o}{=} \PY{n}{l2}\PY{p}{,} 
             \PY{n}{r} \PY{o}{=} \PY{n}{r}\PY{p}{,} \PY{n}{g} \PY{o}{=} \PY{n}{g}\PY{p}{,} \PY{n}{input\PYZus{}max} \PY{o}{=} \PY{n}{input\PYZus{}max}\PY{p}{)}
         
         \PY{l+s+sd}{\PYZsq{}\PYZsq{}\PYZsq{}}
         \PY{l+s+sd}{Code submission for 3.1: }
         \PY{l+s+sd}{Edit this method in `inertial\PYZus{}wheel\PYZus{}pendulum.py`}
         \PY{l+s+sd}{and ensure it produces reasonable A and B}
         \PY{l+s+sd}{\PYZsq{}\PYZsq{}\PYZsq{}}
         \PY{n}{A}\PY{p}{,} \PY{n}{B} \PY{o}{=} \PY{n}{pendulum\PYZus{}plant}\PY{o}{.}\PY{n}{GetLinearizedDynamics}\PY{p}{(}\PY{n}{uf}\PY{p}{,} \PY{n}{xf}\PY{p}{)}
         \PY{k}{print}\PY{p}{(}\PY{l+s+s2}{\PYZdq{}}\PY{l+s+s2}{A: }\PY{l+s+s2}{\PYZdq{}}\PY{p}{,} \PY{n}{A}\PY{p}{)}
         \PY{k}{print}\PY{p}{(}\PY{l+s+s2}{\PYZdq{}}\PY{l+s+s2}{B: }\PY{l+s+s2}{\PYZdq{}}\PY{p}{,} \PY{n}{B}\PY{p}{)}
\end{Verbatim}


    \begin{Verbatim}[commandchars=\\\{\}]
The autoreload extension is already loaded. To reload it, use:
  \%reload\_ext autoreload
[[ 50.   0.]
 [  0.   0.]]
('A: ', array([[ 0.,  0.,  1.,  0.],
       [ 0.,  0.,  0.,  1.],
       [ 5.,  0.,  0.,  0.],
       [-5.,  0.,  0.,  0.]]))
('B: ', array([[ 0. ],
       [ 0. ],
       [-0.1],
       [ 1.1]]))

    \end{Verbatim}

    \subsection{3.2 Controllability (3 points, 2/3
autograded)}\label{controllability-3-points-23-autograded}

\emph{Controllability} is an important propery of a linear system -\/-
you can read about it in Chapter 3 of the textbook.

\textbf{Write a function to test whether a linear system is
controllable, using the function signature below.} (You'll probably want
to do this in two steps: building the controllability matrix, and then
checking its rank.)

\textbf{Is the linearization around the upright controllable? How about
the linearization around the fixed point at \(\theta = 0\)?}

    \begin{Verbatim}[commandchars=\\\{\}]
{\color{incolor}In [{\color{incolor}371}]:} \PY{k}{def} \PY{n+nf}{is\PYZus{}controllable}\PY{p}{(}\PY{n}{A}\PY{p}{,} \PY{n}{B}\PY{p}{)}\PY{p}{:}
              \PY{n}{n} \PY{o}{=} \PY{n}{A}\PY{o}{.}\PY{n}{shape}\PY{p}{[}\PY{l+m+mi}{0}\PY{p}{]}
              \PY{k}{assert} \PY{n}{n} \PY{o}{==} \PY{n}{A}\PY{o}{.}\PY{n}{shape}\PY{p}{[}\PY{l+m+mi}{1}\PY{p}{]}
              \PY{n}{m} \PY{o}{=} \PY{n}{B}\PY{o}{.}\PY{n}{shape}\PY{p}{[}\PY{l+m+mi}{1}\PY{p}{]}
              
              \PY{n}{cont} \PY{o}{=} \PY{n}{np}\PY{o}{.}\PY{n}{zeros}\PY{p}{(}\PY{p}{(}\PY{n}{n}\PY{p}{,} \PY{n}{n}\PY{o}{*}\PY{n}{m}\PY{p}{)}\PY{p}{)}
              \PY{n}{sub} \PY{o}{=} \PY{n}{np}\PY{o}{.}\PY{n}{eye}\PY{p}{(}\PY{n}{A}\PY{o}{.}\PY{n}{shape}\PY{p}{[}\PY{l+m+mi}{0}\PY{p}{]}\PY{p}{)}
              \PY{k}{for} \PY{n}{i} \PY{o+ow}{in} \PY{n+nb}{range}\PY{p}{(}\PY{n}{n}\PY{p}{)}\PY{p}{:}
                  \PY{n}{cont}\PY{p}{[}\PY{p}{:}\PY{p}{,}\PY{n}{m}\PY{o}{*}\PY{n}{i}\PY{p}{:}\PY{n}{m}\PY{o}{*}\PY{p}{(}\PY{n}{i}\PY{o}{+}\PY{l+m+mi}{1}\PY{p}{)}\PY{p}{]} \PY{o}{=} \PY{n}{sub}\PY{o}{.}\PY{n}{dot}\PY{p}{(}\PY{n}{B}\PY{p}{)}
                  \PY{n}{sub} \PY{o}{=} \PY{n}{sub}\PY{o}{.}\PY{n}{dot}\PY{p}{(}\PY{n}{A}\PY{p}{)}
              \PY{k}{return} \PY{n}{np}\PY{o}{.}\PY{n}{linalg}\PY{o}{.}\PY{n}{matrix\PYZus{}rank}\PY{p}{(}\PY{n}{cont}\PY{p}{)} \PY{o}{==} \PY{n}{n}
          
          \PY{c+c1}{\PYZsh{} Play around with using the function}
          \PY{k}{print} \PY{l+s+s2}{\PYZdq{}}\PY{l+s+s2}{A: }\PY{l+s+s2}{\PYZdq{}}\PY{p}{,} \PY{n}{A}
          \PY{k}{print} \PY{l+s+s2}{\PYZdq{}}\PY{l+s+s2}{B: }\PY{l+s+s2}{\PYZdq{}}\PY{p}{,} \PY{n}{B} 
          \PY{k}{print} \PY{l+s+s2}{\PYZdq{}}\PY{l+s+s2}{Is controllable? }\PY{l+s+s2}{\PYZdq{}}\PY{p}{,} \PY{n}{is\PYZus{}controllable}\PY{p}{(}\PY{n}{A}\PY{p}{,} \PY{n}{B}\PY{p}{)}
\end{Verbatim}


    \begin{Verbatim}[commandchars=\\\{\}]
A:  [[ 0.  0.  1.  0.]
 [ 0.  0.  0.  1.]
 [ 5.  0.  0.  0.]
 [-5.  0.  0.  0.]]
B:  [[ 0. ]
 [ 0. ]
 [-0.1]
 [ 1.1]]
Is controllable?  True

    \end{Verbatim}

    \textbf{Short answer explanation for 3.2. Is the linearization around
the upright controllable? How about the linearization around the fixed
point at \(\theta = 0\)?}

Yes, linearization around the upright fixed point is controllable (as
expected), and likewise for the other fixed point at 0.

    \subsection{3.3 LQR (4 points, 4/4
autograded)}\label{lqr-4-points-44-autograded}

Now that we have a linear system to control, let's try employing LQR to
balance the inertial pendulum. Recalling LQR as discussed in class, you
need to supply LQR with a couple things:

\begin{itemize}
\tightlist
\item
  Linear system matrices \(A\) and \(B\)
\item
  Symmetric positive semi-definite state cost matrix \(Q\)
\item
  Symmetric positive definitive input cost matrix \(R\)
\item
  Another cost matrix \(N\) for the off-diagonal terms. (For most of our
  problems, this is set to zero.)
\end{itemize}

And it produces an optimal gain matrix \(K\) for the optimal controller
\(\bar{u} = - K\bar{x}\) that minimizes

\[\min_u \int_0^\infty \bar{x}'Q\bar{x} + \bar{u}'R\bar{u} + 2\bar{x}'N\bar{u} dt \]

LQR also returns the cost-to-go matrix \(S\), with the optimal
cost-to-go taking the form \(J(x) = \bar{x}^T S \bar{x}\).

In this problem, we have a complication: we probably don't want to try
to control the reaction wheel angle \(\theta_2\), as it has no impact on
our dynamics. (Check the dynamics as derived above to see this
yourself!) Your first instinct might be to set the 2nd row and column of
\(Q\) to zero: but unfortunately, this doesn't work, as exemplified
below.

    \begin{Verbatim}[commandchars=\\\{\}]
{\color{incolor}In [{\color{incolor}52}]:} \PY{k+kn}{from} \PY{n+nn}{pydrake.all} \PY{k+kn}{import} \PY{n}{LinearQuadraticRegulator}
         
         \PY{k}{def} \PY{n+nf}{create\PYZus{}lqr}\PY{p}{(}\PY{n}{A}\PY{p}{,} \PY{n}{B}\PY{p}{)}\PY{p}{:}
             \PY{n}{Q} \PY{o}{=} \PY{n}{np}\PY{o}{.}\PY{n}{zeros}\PY{p}{(}\PY{p}{(}\PY{l+m+mi}{4}\PY{p}{,} \PY{l+m+mi}{4}\PY{p}{)}\PY{p}{)}
             \PY{c+c1}{\PYZsh{} Not clear what these gains will do,}
             \PY{c+c1}{\PYZsh{} but as long as Q is positive semidefinite}
             \PY{c+c1}{\PYZsh{} this should find a solution.}
             \PY{n}{Q} \PY{o}{=} \PY{n}{np}\PY{o}{.}\PY{n}{random}\PY{o}{.}\PY{n}{random}\PY{p}{(}\PY{p}{(}\PY{l+m+mi}{4}\PY{p}{,} \PY{l+m+mi}{4}\PY{p}{)}\PY{p}{)}
             \PY{n}{Q} \PY{o}{=} \PY{n}{Q}\PY{o}{.}\PY{n}{T} \PY{o}{+} \PY{n}{Q} \PY{c+c1}{\PYZsh{} make symmetric and thus psd}
             \PY{n}{Q}\PY{p}{[}\PY{p}{:}\PY{p}{,} \PY{l+m+mi}{1}\PY{p}{]} \PY{o}{=} \PY{l+m+mi}{0} \PY{c+c1}{\PYZsh{} Don\PYZsq{}t penalize reaction wheel angle at all}
             \PY{n}{Q}\PY{p}{[}\PY{l+m+mi}{1}\PY{p}{,} \PY{p}{:}\PY{p}{]} \PY{o}{=} \PY{l+m+mi}{0}
             \PY{n}{R} \PY{o}{=} \PY{p}{[}\PY{l+m+mf}{1.}\PY{p}{]}
             \PY{n}{K}\PY{p}{,} \PY{n}{S} \PY{o}{=} \PY{n}{LinearQuadraticRegulator}\PY{p}{(}\PY{n}{A}\PY{p}{,} \PY{n}{B}\PY{p}{,} \PY{n}{Q}\PY{p}{,} \PY{n}{R}\PY{p}{)}
             \PY{k}{return} \PY{p}{(}\PY{n}{K}\PY{p}{,} \PY{n}{S}\PY{p}{)}
         
         \PY{n}{K}\PY{p}{,} \PY{n}{S} \PY{o}{=} \PY{n}{create\PYZus{}lqr}\PY{p}{(}\PY{n}{A}\PY{p}{,} \PY{n}{B}\PY{p}{)}
         \PY{k}{print} \PY{l+s+s2}{\PYZdq{}}\PY{l+s+s2}{LQR K: }\PY{l+s+s2}{\PYZdq{}}\PY{p}{,} \PY{n}{K}
         \PY{k}{if} \PY{n}{np}\PY{o}{.}\PY{n}{any}\PY{p}{(}\PY{n}{np}\PY{o}{.}\PY{n}{isnan}\PY{p}{(}\PY{n}{K}\PY{p}{)}\PY{p}{)}\PY{p}{:}
             \PY{k}{print}\PY{p}{(}\PY{l+s+s2}{\PYZdq{}}\PY{l+s+s2}{Oh no!}\PY{l+s+s2}{\PYZdq{}}\PY{p}{)}
\end{Verbatim}


    \begin{Verbatim}[commandchars=\\\{\}]
LQR K:  [[ nan  nan  nan  nan]]
Oh no!

    \end{Verbatim}

    In this case, the solution is underdetermined (the algebraic Riccati
equation no longer has a \emph{unique} solution, but instead an infinite
number of equally-good ones). Instead, because \(\theta_2\) never
appears in the dynamics, you can form a modified linear system to give
to LQR:

\[\dot{x}_{reduced} = A_{reduced}x_{reduced} + B_{reduced}u_{reduced}\]

where
\(x_{reduced} = \left[\theta_1, \dot \theta_1, \dot \theta_2 \right]^T\),
and insert zeros approximately to bring your final \(K\) and \(S\) back
up to \(1 \times 4\) and \(4 \times 4\) matrices.

\textbf{Do this state reduction, pick reasonable Q and R matrices,
invoke LQR, and reverse your state reduction to generate K and S
matrices.}

    \begin{Verbatim}[commandchars=\\\{\}]
{\color{incolor}In [{\color{incolor}84}]:} \PY{k+kn}{from} \PY{n+nn}{pydrake.all} \PY{k+kn}{import} \PY{p}{(}\PY{n}{BasicVector}\PY{p}{,} \PY{n}{DiagramBuilder}\PY{p}{,} \PY{n}{FloatingBaseType}\PY{p}{,}
                                  \PY{n}{LinearQuadraticRegulator}\PY{p}{,} \PY{n}{RigidBodyPlant}\PY{p}{,}
                                  \PY{n}{RigidBodyTree}\PY{p}{,} \PY{n}{Simulator}\PY{p}{)}
         
         \PY{k}{def} \PY{n+nf}{create\PYZus{}reduced\PYZus{}lqr}\PY{p}{(}\PY{n}{A}\PY{p}{,} \PY{n}{B}\PY{p}{)}\PY{p}{:}
             \PY{l+s+sd}{\PYZsq{}\PYZsq{}\PYZsq{}}
         \PY{l+s+sd}{    Code submission for 3.3: Fill in the missing}
         \PY{l+s+sd}{    details of this function to produce a control}
         \PY{l+s+sd}{    matrix K, and cost\PYZhy{}to\PYZhy{}go matrix S, for the full}
         \PY{l+s+sd}{    (4\PYZhy{}state) system.}
         \PY{l+s+sd}{    \PYZsq{}\PYZsq{}\PYZsq{}}
             
             \PY{n}{Q} \PY{o}{=} \PY{n}{np}\PY{o}{.}\PY{n}{eye}\PY{p}{(}\PY{l+m+mi}{3}\PY{p}{)}
             \PY{n}{R} \PY{o}{=} \PY{p}{[}\PY{l+m+mf}{1.}\PY{p}{]}
             \PY{n}{A1} \PY{o}{=} \PY{n}{np}\PY{o}{.}\PY{n}{delete}\PY{p}{(}\PY{n}{np}\PY{o}{.}\PY{n}{delete}\PY{p}{(}\PY{n}{A}\PY{p}{,} \PY{l+m+mi}{1}\PY{p}{,} \PY{n}{axis}\PY{o}{=}\PY{l+m+mi}{0}\PY{p}{)}\PY{p}{,} \PY{l+m+mi}{1}\PY{p}{,} \PY{n}{axis}\PY{o}{=}\PY{l+m+mi}{1}\PY{p}{)}
             \PY{n}{B1} \PY{o}{=} \PY{n}{np}\PY{o}{.}\PY{n}{delete}\PY{p}{(}\PY{n}{B}\PY{p}{,} \PY{l+m+mi}{1}\PY{p}{,} \PY{n}{axis}\PY{o}{=}\PY{l+m+mi}{0}\PY{p}{)}
             \PY{n}{K}\PY{p}{,} \PY{n}{S} \PY{o}{=} \PY{n}{LinearQuadraticRegulator}\PY{p}{(}\PY{n}{A1}\PY{p}{,} \PY{n}{B1}\PY{p}{,} \PY{n}{Q}\PY{p}{,} \PY{n}{R}\PY{p}{)}
             
             \PY{n}{K1} \PY{o}{=} \PY{n}{np}\PY{o}{.}\PY{n}{insert}\PY{p}{(}\PY{n}{K}\PY{p}{,} \PY{l+m+mi}{1}\PY{p}{,} \PY{n}{axis}\PY{o}{=}\PY{l+m+mi}{1}\PY{p}{,} \PY{n}{values}\PY{o}{=}\PY{l+m+mi}{0}\PY{p}{)}
             \PY{n}{S1} \PY{o}{=} \PY{n}{np}\PY{o}{.}\PY{n}{insert}\PY{p}{(}\PY{n}{S}\PY{p}{,} \PY{l+m+mi}{1}\PY{p}{,} \PY{n}{axis}\PY{o}{=}\PY{l+m+mi}{0}\PY{p}{,} \PY{n}{values}\PY{o}{=}\PY{n}{np}\PY{o}{.}\PY{n}{zeros}\PY{p}{(}\PY{n}{A}\PY{o}{.}\PY{n}{shape}\PY{p}{[}\PY{l+m+mi}{0}\PY{p}{]}\PY{o}{\PYZhy{}}\PY{l+m+mi}{1}\PY{p}{)}\PY{p}{)}
             \PY{n}{S1} \PY{o}{=} \PY{n}{np}\PY{o}{.}\PY{n}{insert}\PY{p}{(}\PY{n}{S1}\PY{p}{,} \PY{l+m+mi}{1}\PY{p}{,} \PY{n}{axis}\PY{o}{=}\PY{l+m+mi}{1}\PY{p}{,} \PY{n}{values}\PY{o}{=}\PY{n}{np}\PY{o}{.}\PY{n}{zeros}\PY{p}{(}\PY{n}{A}\PY{o}{.}\PY{n}{shape}\PY{p}{[}\PY{l+m+mi}{0}\PY{p}{]}\PY{p}{)}\PY{p}{)}
             \PY{k}{return} \PY{p}{(}\PY{n}{K1}\PY{p}{,} \PY{n}{S1}\PY{p}{)}
             
             \PY{c+c1}{\PYZsh{} Refer to create\PYZus{}lqr() to see how invocations}
             \PY{c+c1}{\PYZsh{} to LinearQuadraticRegulator work.}
             \PY{c+c1}{\PYZsh{} https://docs.scipy.org/doc/numpy/reference/generated/numpy.delete.html}
             \PY{c+c1}{\PYZsh{} and }
             \PY{c+c1}{\PYZsh{} https://docs.scipy.org/doc/numpy/reference/generated/numpy.insert.html}
             \PY{c+c1}{\PYZsh{} might be useful for helping with the state reduction.}
             
             \PY{k}{return} \PY{p}{(}\PY{n}{K}\PY{p}{,} \PY{n}{S}\PY{p}{)}
         
         \PY{n}{K}\PY{p}{,} \PY{n}{S} \PY{o}{=} \PY{n}{create\PYZus{}reduced\PYZus{}lqr}\PY{p}{(}\PY{n}{A}\PY{p}{,} \PY{n}{B}\PY{p}{)}
         \PY{k}{print} \PY{l+s+s2}{\PYZdq{}}\PY{l+s+s2}{K: }\PY{l+s+s2}{\PYZdq{}}\PY{p}{,} \PY{n}{K}
         \PY{k}{print} \PY{l+s+s2}{\PYZdq{}}\PY{l+s+s2}{S: }\PY{l+s+s2}{\PYZdq{}}\PY{p}{,} \PY{n}{S}
\end{Verbatim}


    \begin{Verbatim}[commandchars=\\\{\}]
K:  [[-145.425    0.     -66.05    -1.   ]]
S:  [[ 4753.172     0.     2180.804    66.05 ]
 [    0.        0.        0.        0.   ]
 [ 2180.804     0.     1001.43     30.994]
 [   66.05      0.       30.994     1.909]]

    \end{Verbatim}

    \textbf{Finally, use this controller to stabilize the robot in
simulation!}

    \begin{Verbatim}[commandchars=\\\{\}]
{\color{incolor}In [{\color{incolor}333}]:} \PY{k+kn}{from} \PY{n+nn}{inertial\PYZus{}wheel\PYZus{}pendulum} \PY{k+kn}{import} \PY{o}{*}
          \PY{k+kn}{from} \PY{n+nn}{IPython.display} \PY{k+kn}{import} \PY{n}{HTML}
          \PY{k+kn}{from} \PY{n+nn}{inertial\PYZus{}wheel\PYZus{}pendulum\PYZus{}visualizer} \PY{k+kn}{import} \PY{o}{*}
          \PY{k+kn}{import} \PY{n+nn}{matplotlib.pyplot} \PY{k+kn}{as} \PY{n+nn}{plt}
          
          \PY{k}{def} \PY{n+nf}{lqr\PYZus{}controller}\PY{p}{(}\PY{n}{x}\PY{p}{)}\PY{p}{:}
              \PY{c+c1}{\PYZsh{} This should return a 1x1 u that is bounded}
              \PY{c+c1}{\PYZsh{} between \PYZhy{}input\PYZus{}max and input\PYZus{}max.}
              \PY{c+c1}{\PYZsh{} Remember to wrap the angular values back to}
              \PY{c+c1}{\PYZsh{} [\PYZhy{}pi, pi].}
              \PY{n}{u} \PY{o}{=} \PY{n}{np}\PY{o}{.}\PY{n}{zeros}\PY{p}{(}\PY{p}{(}\PY{l+m+mi}{1}\PY{p}{,} \PY{l+m+mi}{1}\PY{p}{)}\PY{p}{)}
              \PY{k}{global} \PY{n}{xf}\PY{p}{,} \PY{n}{uf}\PY{p}{,} \PY{n}{K}
              \PY{n}{x} \PY{o}{=} \PY{n}{x} \PY{o}{\PYZhy{}} \PY{n}{xf}    
              \PY{n}{x}\PY{p}{[}\PY{l+m+mi}{0}\PY{p}{]} \PY{o}{=} \PY{p}{(}\PY{n}{x}\PY{p}{[}\PY{l+m+mi}{0}\PY{p}{]} \PY{o}{+} \PY{n}{np}\PY{o}{.}\PY{n}{pi}\PY{p}{)} \PY{o}{\PYZpc{}} \PY{p}{(}\PY{l+m+mi}{2} \PY{o}{*} \PY{n}{np}\PY{o}{.}\PY{n}{pi} \PY{p}{)} \PY{o}{\PYZhy{}} \PY{n}{np}\PY{o}{.}\PY{n}{pi}
              \PY{n}{x}\PY{p}{[}\PY{l+m+mi}{1}\PY{p}{]} \PY{o}{=} \PY{p}{(}\PY{n}{x}\PY{p}{[}\PY{l+m+mi}{1}\PY{p}{]} \PY{o}{+} \PY{n}{np}\PY{o}{.}\PY{n}{pi}\PY{p}{)} \PY{o}{\PYZpc{}} \PY{p}{(}\PY{l+m+mi}{2} \PY{o}{*} \PY{n}{np}\PY{o}{.}\PY{n}{pi} \PY{p}{)} \PY{o}{\PYZhy{}} \PY{n}{np}\PY{o}{.}\PY{n}{pi}
              \PY{n}{u} \PY{o}{=} \PY{n}{np}\PY{o}{.}\PY{n}{clip}\PY{p}{(}\PY{n}{uf} \PY{o}{\PYZhy{}} \PY{n}{K}\PY{o}{.}\PY{n}{dot}\PY{p}{(} \PY{n}{x} \PY{p}{)}\PY{p}{,} \PY{o}{\PYZhy{}}\PY{n}{input\PYZus{}max}\PY{p}{,} \PY{n}{input\PYZus{}max}\PY{p}{)}
              
              \PY{k}{return} \PY{n}{u}
          
          \PY{c+c1}{\PYZsh{} Run forward simulation from the specified initial condition}
          \PY{n}{duration} \PY{o}{=} \PY{l+m+mf}{2.}
          \PY{c+c1}{\PYZsh{}x0 = [3, 0.0, 0.0, 5.0]}
          \PY{n}{x0} \PY{o}{=} \PY{p}{[}\PY{l+m+mi}{3}\PY{o}{*}\PY{n}{np}\PY{o}{.}\PY{n}{pi}\PY{o}{+}\PY{l+m+mf}{0.1}\PY{p}{,} \PY{l+m+mf}{0.0}\PY{p}{,} \PY{l+m+mf}{0.0}\PY{p}{,} \PY{l+m+mf}{5.0}\PY{p}{]}
          
          \PY{n}{input\PYZus{}log}\PY{p}{,} \PY{n}{state\PYZus{}log} \PY{o}{=} \PYZbs{}
              \PY{n}{RunSimulation}\PY{p}{(}\PY{n}{pendulum\PYZus{}plant}\PY{p}{,}
                        \PY{n}{lqr\PYZus{}controller}\PY{p}{,}
                        \PY{n}{x0}\PY{o}{=}\PY{n}{x0}\PY{p}{,}
                        \PY{n}{duration}\PY{o}{=}\PY{n}{duration}\PY{p}{)}
          
              
          \PY{c+c1}{\PYZsh{} Visualize state and input traces}
          \PY{n}{fig} \PY{o}{=} \PY{n}{plt}\PY{o}{.}\PY{n}{figure}\PY{p}{(}\PY{p}{)}\PY{o}{.}\PY{n}{set\PYZus{}size\PYZus{}inches}\PY{p}{(}\PY{l+m+mi}{6}\PY{p}{,} \PY{l+m+mi}{6}\PY{p}{)}
          \PY{k}{for} \PY{n}{i} \PY{o+ow}{in} \PY{n+nb}{range}\PY{p}{(}\PY{l+m+mi}{4}\PY{p}{)}\PY{p}{:}
              \PY{n}{plt}\PY{o}{.}\PY{n}{subplot}\PY{p}{(}\PY{l+m+mi}{5}\PY{p}{,} \PY{l+m+mi}{1}\PY{p}{,} \PY{n}{i}\PY{o}{+}\PY{l+m+mi}{1}\PY{p}{)}
              \PY{n}{plt}\PY{o}{.}\PY{n}{plot}\PY{p}{(}\PY{n}{state\PYZus{}log}\PY{o}{.}\PY{n}{sample\PYZus{}times}\PY{p}{(}\PY{p}{)}\PY{p}{,} \PY{n}{state\PYZus{}log}\PY{o}{.}\PY{n}{data}\PY{p}{(}\PY{p}{)}\PY{p}{[}\PY{n}{i}\PY{p}{,} \PY{p}{:}\PY{p}{]}\PY{p}{)}
              \PY{n}{plt}\PY{o}{.}\PY{n}{grid}\PY{p}{(}\PY{n+nb+bp}{True}\PY{p}{)}
              \PY{n}{plt}\PY{o}{.}\PY{n}{ylabel}\PY{p}{(}\PY{l+s+s2}{\PYZdq{}}\PY{l+s+s2}{x[}\PY{l+s+si}{\PYZpc{}d}\PY{l+s+s2}{]}\PY{l+s+s2}{\PYZdq{}} \PY{o}{\PYZpc{}} \PY{n}{i}\PY{p}{)}
          \PY{n}{plt}\PY{o}{.}\PY{n}{subplot}\PY{p}{(}\PY{l+m+mi}{5}\PY{p}{,} \PY{l+m+mi}{1}\PY{p}{,} \PY{l+m+mi}{5}\PY{p}{)}
          \PY{n}{plt}\PY{o}{.}\PY{n}{plot}\PY{p}{(}\PY{n}{input\PYZus{}log}\PY{o}{.}\PY{n}{sample\PYZus{}times}\PY{p}{(}\PY{p}{)}\PY{p}{,} \PY{n}{input\PYZus{}log}\PY{o}{.}\PY{n}{data}\PY{p}{(}\PY{p}{)}\PY{p}{[}\PY{l+m+mi}{0}\PY{p}{,} \PY{p}{:}\PY{p}{]}\PY{p}{)}
          \PY{n}{plt}\PY{o}{.}\PY{n}{ylabel}\PY{p}{(}\PY{l+s+s2}{\PYZdq{}}\PY{l+s+s2}{u[0]}\PY{l+s+s2}{\PYZdq{}}\PY{p}{)}
          \PY{n}{plt}\PY{o}{.}\PY{n}{xlabel}\PY{p}{(}\PY{l+s+s2}{\PYZdq{}}\PY{l+s+s2}{t}\PY{l+s+s2}{\PYZdq{}}\PY{p}{)}
          \PY{n}{plt}\PY{o}{.}\PY{n}{grid}\PY{p}{(}\PY{n+nb+bp}{True}\PY{p}{)}
          
          \PY{c+c1}{\PYZsh{} Visualize the simulation}
          \PY{n}{viz} \PY{o}{=} \PY{n}{InertialWheelPendulumVisualizer}\PY{p}{(}\PY{n}{pendulum\PYZus{}plant}\PY{p}{)}
          \PY{n}{ani} \PY{o}{=} \PY{n}{viz}\PY{o}{.}\PY{n}{animate}\PY{p}{(}\PY{n}{input\PYZus{}log}\PY{p}{,} \PY{n}{state\PYZus{}log}\PY{p}{,} \PY{l+m+mi}{30}\PY{p}{,} \PY{n}{repeat}\PY{o}{=}\PY{n+nb+bp}{True}\PY{p}{)}
          \PY{n}{plt}\PY{o}{.}\PY{n}{close}\PY{p}{(}\PY{n}{viz}\PY{o}{.}\PY{n}{fig}\PY{p}{)}
          \PY{n}{HTML}\PY{p}{(}\PY{n}{ani}\PY{o}{.}\PY{n}{to\PYZus{}html5\PYZus{}video}\PY{p}{(}\PY{p}{)}\PY{p}{)}
\end{Verbatim}


\begin{Verbatim}[commandchars=\\\{\}]
{\color{outcolor}Out[{\color{outcolor}333}]:} <IPython.core.display.HTML object>
\end{Verbatim}
            
    \begin{center}
    \adjustimage{max size={0.9\linewidth}{0.9\paperheight}}{output_35_1.png}
    \end{center}
    { \hspace*{\fill} \\}
    
    \subsection{3.4 LQR Region of Attraction, Prologue (2 points, 2/2
autograded)}\label{lqr-region-of-attraction-prologue-2-points-22-autograded}

As you probably found when playing with the LQR simulation in the
previous question, there are plenty of states for which the LQR doesn't
converge. \textbf{Find an x0 from which LQR does \emph{not} converge,
and an x0 from which is does, and write them below.}

    \begin{Verbatim}[commandchars=\\\{\}]
{\color{incolor}In [{\color{incolor}89}]:} \PY{l+s+sd}{\PYZsq{}\PYZsq{}\PYZsq{}}
         \PY{l+s+sd}{Code submission for 3.4}
         \PY{l+s+sd}{\PYZsq{}\PYZsq{}\PYZsq{}}
         \PY{k}{def} \PY{n+nf}{get\PYZus{}x0\PYZus{}does\PYZus{}not\PYZus{}converge}\PY{p}{(}\PY{p}{)}\PY{p}{:}
             \PY{k}{return} \PY{p}{[}\PY{l+m+mi}{1}\PY{p}{,} \PY{l+m+mf}{0.0}\PY{p}{,} \PY{l+m+mf}{0.0}\PY{p}{,} \PY{l+m+mf}{0.0}\PY{p}{]}
         \PY{k}{def} \PY{n+nf}{get\PYZus{}x0\PYZus{}does\PYZus{}converge}\PY{p}{(}\PY{p}{)}\PY{p}{:}
             \PY{k}{return} \PY{p}{[}\PY{n}{math}\PY{o}{.}\PY{n}{pi}\PY{p}{,} \PY{l+m+mf}{0.0}\PY{p}{,} \PY{l+m+mf}{0.0}\PY{p}{,} \PY{l+m+mf}{0.0}\PY{p}{]}
\end{Verbatim}


    \subsection{3.5 LQR Region of Attraction, Episode 1 (5 points, 2/5
autograded)}\label{lqr-region-of-attraction-episode-1-5-points-25-autograded}

What we would like to know is when the LQR controller will work, and
when it won't. Let's see if we can analyze the region of attraction of
this controller using the tools of Lyapunov stability analysis.

The closed-loop system dynamics of our system when using the LQR
controller are

\[ \dot{x} = f(x, -K(x-x_f)) \]

A function that we know is a decent Lyapunov function near \(x_f\) is
the cost-to-go of the LQR solution: \(V = \bar{x}^T S \bar{x}\).

Following section
\href{http://underactuated.csail.mit.edu/underactuated.html?chapter=lyapunov}{10.3
of the textbook}, we just need to find a \(\rho\) such that

\[ 
\dot{V}(x) \prec 0, \ \ \  \forall x : V(x) > 0
\]

to demonstrate the running LQR on our full (nonlinear!) system from
state \(x_0\) will converge as long as \(V(x_0) \leq \rho\).

Let's start by calculating \(\dot{V}(x)\) and \(\dot{V}(x)\) for the
full system. \textbf{Using the helper function \emph{evaluate\_f(x, u)}
(it's a member of InertialWheelPendulum in
\emph{inertial\_wheel\_pendulum.py}) to fill out \emph{calcV} and
\emph{calcVdot} to do this.}

Running this cell will populate the variables \emph{V\_samples},
\emph{f\_samples}, and \emph{Vdot\_samples} in the plane where
\(\theta_2 = 0\). Feel free to tweak those the sample grid (defined by
n\_bins, theta\_widthm, etc) as you wish, but we'll be using these
samples to estimate the region of attraction, so keep the resolution as
high as you can while keeping the runtime reasonable. (We are,
unfortunately, sampling in 3 dimensions...)

    \begin{Verbatim}[commandchars=\\\{\}]
{\color{incolor}In [{\color{incolor}334}]:} \PY{k+kn}{import} \PY{n+nn}{matplotlib.pyplot} \PY{k+kn}{as} \PY{n+nn}{plt}
          \PY{k+kn}{import} \PY{n+nn}{time}
          
          \PY{c+c1}{\PYZsh{} Calculates the closed loop f(x) at xn}
          \PY{k}{def} \PY{n+nf}{calcF}\PY{p}{(}\PY{n}{xn}\PY{p}{)}\PY{p}{:}
              \PY{c+c1}{\PYZsh{} Feel free to bring in whatever}
              \PY{c+c1}{\PYZsh{} global variables you need, e.g.:}
              \PY{n}{xn} \PY{o}{=} \PY{n}{np}\PY{o}{.}\PY{n}{array}\PY{p}{(}\PY{n}{xn}\PY{p}{)}
              \PY{k}{global} \PY{n}{xf}
              \PY{k}{global} \PY{n}{pendulum\PYZus{}plant}
              \PY{k}{global} \PY{n}{K}
              \PY{n}{xbar} \PY{o}{=} \PY{n}{xn} \PY{o}{\PYZhy{}} \PY{n}{xf}
              \PY{n}{u} \PY{o}{=} \PY{o}{\PYZhy{}}\PY{n}{K}\PY{o}{.}\PY{n}{dot}\PY{p}{(}\PY{n}{xbar}\PY{p}{)} \PY{o}{+} \PY{n}{uf}
              \PY{k}{return} \PY{n}{pendulum\PYZus{}plant}\PY{o}{.}\PY{n}{evaluate\PYZus{}f}\PY{p}{(}\PY{n}{u}\PY{p}{,} \PY{n}{xn}\PY{p}{,} \PY{n}{throw\PYZus{}when\PYZus{}limits\PYZus{}exceeded}\PY{o}{=}\PY{n+nb+bp}{False}\PY{p}{)}
          
          \PY{c+c1}{\PYZsh{} Calculates V(xn)}
          \PY{k}{def} \PY{n+nf}{calcV}\PY{p}{(}\PY{n}{xn}\PY{p}{)}\PY{p}{:}
              \PY{c+c1}{\PYZsh{} Feel free to bring in whatever}
              \PY{c+c1}{\PYZsh{} global variables you need}
              \PY{k}{global} \PY{n}{K}\PY{p}{,} \PY{n}{S}
              \PY{n}{xn}\PY{p}{[}\PY{l+m+mi}{0}\PY{p}{]} \PY{o}{=} \PY{p}{(}\PY{n}{xn}\PY{p}{[}\PY{l+m+mi}{0}\PY{p}{]} \PY{o}{\PYZpc{}} \PY{p}{(}\PY{l+m+mi}{2}\PY{o}{*}\PY{n}{np}\PY{o}{.}\PY{n}{pi}\PY{p}{)} \PY{o}{+} \PY{l+m+mi}{2}\PY{o}{*}\PY{n}{np}\PY{o}{.}\PY{n}{pi}\PY{p}{)} \PY{o}{\PYZpc{}} \PY{p}{(}\PY{l+m+mi}{2}\PY{o}{*}\PY{n}{np}\PY{o}{.}\PY{n}{pi}\PY{p}{)}
              \PY{n}{xn}\PY{p}{[}\PY{l+m+mi}{1}\PY{p}{]} \PY{o}{=} \PY{p}{(}\PY{n}{xn}\PY{p}{[}\PY{l+m+mi}{1}\PY{p}{]} \PY{o}{\PYZpc{}} \PY{p}{(}\PY{l+m+mi}{2}\PY{o}{*}\PY{n}{np}\PY{o}{.}\PY{n}{pi}\PY{p}{)} \PY{o}{+} \PY{l+m+mi}{2}\PY{o}{*}\PY{n}{np}\PY{o}{.}\PY{n}{pi}\PY{p}{)} \PY{o}{\PYZpc{}} \PY{p}{(}\PY{l+m+mi}{2}\PY{o}{*}\PY{n}{np}\PY{o}{.}\PY{n}{pi}\PY{p}{)}
              \PY{n}{xb} \PY{o}{=} \PY{n}{xn} \PY{o}{\PYZhy{}} \PY{n}{xf}
              \PY{k}{return} \PY{n}{xb}\PY{o}{.}\PY{n}{dot}\PY{p}{(}\PY{n}{S}\PY{p}{)}\PY{o}{.}\PY{n}{dot}\PY{p}{(}\PY{n}{xb}\PY{p}{)}
          
          \PY{c+c1}{\PYZsh{} Calculates \PYZbs{}dot\PYZob{}V\PYZcb{}(xn).}
          \PY{k}{def} \PY{n+nf}{calcVdot}\PY{p}{(}\PY{n}{xn}\PY{p}{)}\PY{p}{:}
              \PY{c+c1}{\PYZsh{} Feel free to bring in whatever}
              \PY{c+c1}{\PYZsh{} global variables you need}
              \PY{k}{global} \PY{n}{A}\PY{p}{,} \PY{n}{B}\PY{p}{,} \PY{n}{Q}\PY{p}{,} \PY{n}{R}
              \PY{k}{global} \PY{n}{K}\PY{p}{,} \PY{n}{S}
              \PY{n}{xn}\PY{p}{[}\PY{l+m+mi}{0}\PY{p}{]} \PY{o}{=} \PY{p}{(}\PY{n}{xn}\PY{p}{[}\PY{l+m+mi}{0}\PY{p}{]} \PY{o}{\PYZpc{}} \PY{p}{(}\PY{l+m+mi}{2}\PY{o}{*}\PY{n}{np}\PY{o}{.}\PY{n}{pi}\PY{p}{)} \PY{o}{+} \PY{l+m+mi}{2}\PY{o}{*}\PY{n}{np}\PY{o}{.}\PY{n}{pi}\PY{p}{)} \PY{o}{\PYZpc{}} \PY{p}{(}\PY{l+m+mi}{2}\PY{o}{*}\PY{n}{np}\PY{o}{.}\PY{n}{pi}\PY{p}{)}
              \PY{n}{xn}\PY{p}{[}\PY{l+m+mi}{1}\PY{p}{]} \PY{o}{=} \PY{p}{(}\PY{n}{xn}\PY{p}{[}\PY{l+m+mi}{1}\PY{p}{]} \PY{o}{\PYZpc{}} \PY{p}{(}\PY{l+m+mi}{2}\PY{o}{*}\PY{n}{np}\PY{o}{.}\PY{n}{pi}\PY{p}{)} \PY{o}{+} \PY{l+m+mi}{2}\PY{o}{*}\PY{n}{np}\PY{o}{.}\PY{n}{pi}\PY{p}{)} \PY{o}{\PYZpc{}} \PY{p}{(}\PY{l+m+mi}{2}\PY{o}{*}\PY{n}{np}\PY{o}{.}\PY{n}{pi}\PY{p}{)}
              \PY{n}{xb} \PY{o}{=} \PY{n}{xn} \PY{o}{\PYZhy{}} \PY{n}{xf}
              \PY{k}{return} \PY{l+m+mi}{2} \PY{o}{*} \PY{n}{S}\PY{o}{.}\PY{n}{dot}\PY{p}{(}\PY{n}{xb}\PY{p}{)}\PY{o}{.}\PY{n}{dot}\PY{p}{(}\PY{n}{calcF}\PY{p}{(}\PY{n}{xn}\PY{p}{)}\PY{p}{)}
          
          \PY{n}{start\PYZus{}time} \PY{o}{=} \PY{n}{time}\PY{o}{.}\PY{n}{time}\PY{p}{(}\PY{p}{)}
          
          \PY{c+c1}{\PYZsh{} Sample f, V, and Vdot over}
          \PY{c+c1}{\PYZsh{} a grid defined by these parameters.}
          \PY{c+c1}{\PYZsh{} (Odd numbers are good because there\PYZsq{}ll be}
          \PY{c+c1}{\PYZsh{} a bin at exactly the origin.}
          \PY{c+c1}{\PYZsh{} These are slightly strange numbers as we\PYZsq{}ve}
          \PY{c+c1}{\PYZsh{} tried to default these to something as small}
          \PY{c+c1}{\PYZsh{} as possible while still giving reasonable results.}
          \PY{c+c1}{\PYZsh{} Feel free to increase if your computer and patience}
          \PY{c+c1}{\PYZsh{} can handle it.)}
          \PY{n}{n\PYZus{}bins} \PY{o}{=} \PY{l+m+mi}{41}
          \PY{n}{n\PYZus{}bins\PYZus{}theta2d} \PY{o}{=} \PY{l+m+mi}{7}
          \PY{c+c1}{\PYZsh{} For theta and thetad, we only need to span}
          \PY{c+c1}{\PYZsh{} a small region around the fixed point}
          \PY{n}{theta\PYZus{}width} \PY{o}{=} \PY{l+m+mi}{4}
          \PY{n}{thetad\PYZus{}width} \PY{o}{=} \PY{l+m+mi}{4}
          \PY{c+c1}{\PYZsh{} For \PYZbs{}dot\PYZob{}theta\PYZus{}2\PYZcb{}, though, the default}
          \PY{c+c1}{\PYZsh{} parameters for our pendulum lead us to}
          \PY{c+c1}{\PYZsh{} need to search larger absolute \PYZbs{}dot\PYZob{}theta\PYZus{}2\PYZcb{}}
          \PY{c+c1}{\PYZsh{} values (because the inertial wheel is relatively}
          \PY{c+c1}{\PYZsh{} light).}
          \PY{n}{theta2d\PYZus{}width} \PY{o}{=} \PY{l+m+mi}{50}
          
          \PY{c+c1}{\PYZsh{} Do the actual sampling....}
          \PY{n}{x} \PY{o}{=} \PY{n}{np}\PY{o}{.}\PY{n}{linspace}\PY{p}{(}\PY{n}{xf}\PY{p}{[}\PY{l+m+mi}{0}\PY{p}{]}\PY{o}{\PYZhy{}}\PY{n}{theta\PYZus{}width}\PY{p}{,} \PY{n}{xf}\PY{p}{[}\PY{l+m+mi}{0}\PY{p}{]}\PY{o}{+}\PY{n}{theta\PYZus{}width}\PY{p}{,} \PY{n}{n\PYZus{}bins}\PY{p}{)}
          \PY{n}{y} \PY{o}{=} \PY{n}{np}\PY{o}{.}\PY{n}{linspace}\PY{p}{(}\PY{n}{xf}\PY{p}{[}\PY{l+m+mi}{2}\PY{p}{]}\PY{o}{\PYZhy{}}\PY{n}{thetad\PYZus{}width}\PY{p}{,} \PY{n}{xf}\PY{p}{[}\PY{l+m+mi}{2}\PY{p}{]}\PY{o}{+}\PY{n}{thetad\PYZus{}width}\PY{p}{,} \PY{n}{n\PYZus{}bins}\PY{p}{)}
          \PY{n}{z} \PY{o}{=} \PY{n}{np}\PY{o}{.}\PY{n}{linspace}\PY{p}{(}\PY{n}{xf}\PY{p}{[}\PY{l+m+mi}{3}\PY{p}{]}\PY{o}{\PYZhy{}}\PY{n}{theta2d\PYZus{}width}\PY{p}{,} \PY{n}{xf}\PY{p}{[}\PY{l+m+mi}{3}\PY{p}{]}\PY{o}{+}\PY{n}{theta2d\PYZus{}width}\PY{p}{,} \PY{n}{n\PYZus{}bins\PYZus{}theta2d}\PY{p}{)}
          \PY{n}{X}\PY{p}{,} \PY{n}{Y}\PY{p}{,} \PY{n}{Z} \PY{o}{=} \PY{n}{np}\PY{o}{.}\PY{n}{meshgrid}\PY{p}{(}\PY{n}{x}\PY{p}{,} \PY{n}{y}\PY{p}{,} \PY{n}{z}\PY{p}{,} \PY{n}{indexing}\PY{o}{=}\PY{l+s+s2}{\PYZdq{}}\PY{l+s+s2}{ij}\PY{l+s+s2}{\PYZdq{}}\PY{p}{)}
          
          \PY{k}{def} \PY{n+nf}{calc\PYZus{}over\PYZus{}array}\PY{p}{(}\PY{n}{f}\PY{p}{)}\PY{p}{:}
              \PY{k}{return} \PY{n}{np}\PY{o}{.}\PY{n}{array}\PY{p}{(}\PY{p}{[}\PY{p}{[}\PY{p}{[}\PY{n}{f}\PY{p}{(}\PY{p}{[}\PY{n}{dx}\PY{p}{,} \PY{l+m+mf}{0.}\PY{p}{,} \PY{n}{dy}\PY{p}{,} \PY{n}{dz}\PY{p}{]}\PY{p}{)} \PY{k}{for} \PY{n}{dz} \PY{o+ow}{in} \PY{n}{z}\PY{p}{]} \PY{k}{for} \PY{n}{dx} \PY{o+ow}{in} \PY{n}{x}\PY{p}{]} \PY{k}{for} \PY{n}{dy} \PY{o+ow}{in} \PY{n}{y}\PY{p}{]}\PY{p}{)}
          
          \PY{n}{V\PYZus{}samples} \PY{o}{=} \PY{n}{calc\PYZus{}over\PYZus{}array}\PY{p}{(}\PY{n}{calcV}\PY{p}{)}
          \PY{n}{f\PYZus{}samples} \PY{o}{=} \PY{n}{calc\PYZus{}over\PYZus{}array}\PY{p}{(}\PY{n}{calcF}\PY{p}{)}
          \PY{n}{Vdot\PYZus{}samples} \PY{o}{=} \PY{n}{calc\PYZus{}over\PYZus{}array}\PY{p}{(}\PY{n}{calcVdot}\PY{p}{)}
          
          \PY{n}{elapsed} \PY{o}{=} \PY{n}{time}\PY{o}{.}\PY{n}{time}\PY{p}{(}\PY{p}{)} \PY{o}{\PYZhy{}} \PY{n}{start\PYZus{}time}
          \PY{k}{print} \PY{l+s+s2}{\PYZdq{}}\PY{l+s+s2}{Computed }\PY{l+s+si}{\PYZpc{}d}\PY{l+s+s2}{ x }\PY{l+s+si}{\PYZpc{}d}\PY{l+s+s2}{ x }\PY{l+s+si}{\PYZpc{}d}\PY{l+s+s2}{ sampling in }\PY{l+s+si}{\PYZpc{}f}\PY{l+s+s2}{ seconds}\PY{l+s+s2}{\PYZdq{}} \PY{o}{\PYZpc{}} \PY{p}{(}\PY{n}{n\PYZus{}bins}\PY{p}{,} \PY{n}{n\PYZus{}bins}\PY{p}{,} \PY{n}{n\PYZus{}bins\PYZus{}theta2d}\PY{p}{,} \PY{n}{elapsed}\PY{p}{)}
\end{Verbatim}


    \begin{Verbatim}[commandchars=\\\{\}]
Computed 41 x 41 x 7 sampling in 1.081795 seconds

    \end{Verbatim}

    \begin{Verbatim}[commandchars=\\\{\}]
{\color{incolor}In [{\color{incolor}335}]:} \PY{c+c1}{\PYZsh{} This cell plots the samples using color\PYZhy{}coded plots.}
          \PY{c+c1}{\PYZsh{} Color coding:}
          \PY{c+c1}{\PYZsh{}   V: blue = low\PYZhy{}value, red = high\PYZhy{}value}
          \PY{c+c1}{\PYZsh{}   Vdot: blue = low value, yellow = around 0, red = high value}
          \PY{c+c1}{\PYZsh{} The plot of Vdot is overlayed with a quiver plot of the samples}
          \PY{c+c1}{\PYZsh{} of f.}
          
          \PY{c+c1}{\PYZsh{} Select with slice of \PYZbs{}dot\PYZob{}theta\PYZus{}2\PYZcb{} we\PYZsq{}ll}
          \PY{c+c1}{\PYZsh{} plot... this slice should be close to 0,}
          \PY{c+c1}{\PYZsh{} as it\PYZsq{}s the middle bin.}
          \PY{n}{theta2d\PYZus{}plotting\PYZus{}slice} \PY{o}{=} \PY{n}{n\PYZus{}bins\PYZus{}theta2d} \PY{o}{/} \PY{l+m+mi}{2}
          
          \PY{n}{plt}\PY{o}{.}\PY{n}{figure}\PY{p}{(}\PY{p}{)}\PY{o}{.}\PY{n}{set\PYZus{}size\PYZus{}inches}\PY{p}{(}\PY{l+m+mi}{12}\PY{p}{,}\PY{l+m+mi}{12}\PY{p}{)}
          
          \PY{c+c1}{\PYZsh{} Plot V}
          \PY{n}{Xplot}\PY{p}{,} \PY{n}{Yplot} \PY{o}{=} \PY{n}{np}\PY{o}{.}\PY{n}{meshgrid}\PY{p}{(}\PY{n}{x}\PY{p}{,} \PY{n}{y}\PY{p}{)}
          \PY{n}{plt}\PY{o}{.}\PY{n}{subplot}\PY{p}{(}\PY{l+m+mi}{2}\PY{p}{,} \PY{l+m+mi}{1}\PY{p}{,} \PY{l+m+mi}{1}\PY{p}{)}
          \PY{n}{plt}\PY{o}{.}\PY{n}{pcolormesh}\PY{p}{(}\PY{n}{Xplot}\PY{p}{,} \PY{n}{Yplot}\PY{p}{,} \PY{n}{V\PYZus{}samples}\PY{p}{[}\PY{p}{:}\PY{p}{,} \PY{p}{:}\PY{p}{,} \PY{n}{theta2d\PYZus{}plotting\PYZus{}slice}\PY{p}{]}\PY{p}{)}
          \PY{n}{plt}\PY{o}{.}\PY{n}{title}\PY{p}{(}\PY{l+s+s2}{\PYZdq{}}\PY{l+s+s2}{V(x) at dtheta\PYZus{}2 = }\PY{l+s+si}{\PYZpc{}f}\PY{l+s+s2}{\PYZdq{}} \PY{o}{\PYZpc{}} \PY{n}{z}\PY{p}{[}\PY{n}{theta2d\PYZus{}plotting\PYZus{}slice}\PY{p}{]}\PY{p}{)}
          \PY{n}{plt}\PY{o}{.}\PY{n}{xlabel}\PY{p}{(}\PY{l+s+s2}{\PYZdq{}}\PY{l+s+s2}{theta\PYZus{}1}\PY{l+s+s2}{\PYZdq{}}\PY{p}{)}
          \PY{n}{plt}\PY{o}{.}\PY{n}{ylabel}\PY{p}{(}\PY{l+s+s2}{\PYZdq{}}\PY{l+s+s2}{dtheta\PYZus{}1}\PY{l+s+s2}{\PYZdq{}}\PY{p}{)}
          
          \PY{c+c1}{\PYZsh{} Plot Vdot}
          \PY{c+c1}{\PYZsh{} Use a sigmoid to try to squash the huge range of Vdot}
          \PY{c+c1}{\PYZsh{} into something more visually appealing.}
          \PY{k}{def} \PY{n+nf}{sigmoid}\PY{p}{(}\PY{n}{x}\PY{p}{)}\PY{p}{:}
              \PY{k}{return} \PY{l+m+mi}{1} \PY{o}{/} \PY{p}{(}\PY{l+m+mi}{1} \PY{o}{+} \PY{n}{np}\PY{o}{.}\PY{n}{exp}\PY{p}{(}\PY{o}{\PYZhy{}}\PY{n}{x}\PY{o}{/}\PY{l+m+mf}{1000.}\PY{p}{)}\PY{p}{)}
          
          \PY{n}{plt}\PY{o}{.}\PY{n}{subplot}\PY{p}{(}\PY{l+m+mi}{2}\PY{p}{,} \PY{l+m+mi}{1}\PY{p}{,} \PY{l+m+mi}{2}\PY{p}{)}
          \PY{n}{Vdot\PYZus{}viz} \PY{o}{=} \PY{n}{sigmoid}\PY{p}{(}\PY{n}{Vdot\PYZus{}samples}\PY{p}{[}\PY{p}{:}\PY{p}{,} \PY{p}{:}\PY{p}{,} \PY{n}{theta2d\PYZus{}plotting\PYZus{}slice}\PY{p}{]}\PY{p}{)}
          \PY{n}{plt}\PY{o}{.}\PY{n}{pcolormesh}\PY{p}{(}\PY{n}{Xplot}\PY{p}{,} \PY{n}{Yplot}\PY{p}{,} \PY{n}{Vdot\PYZus{}viz}\PY{p}{,} \PY{n}{vmin}\PY{o}{=}\PY{l+m+mi}{0}\PY{p}{,} \PY{n}{vmax}\PY{o}{=}\PY{l+m+mf}{1.0}\PY{p}{)}
          \PY{n}{plt}\PY{o}{.}\PY{n}{title}\PY{p}{(}\PY{l+s+s2}{\PYZdq{}}\PY{l+s+s2}{Vdot(x) overlaid with phase diagram f(x) at dtheta\PYZus{}2 = }\PY{l+s+si}{\PYZpc{}f}\PY{l+s+s2}{\PYZdq{}} \PY{o}{\PYZpc{}} \PY{n}{z}\PY{p}{[}\PY{n}{theta2d\PYZus{}plotting\PYZus{}slice}\PY{p}{]}\PY{p}{)}
          \PY{n}{plt}\PY{o}{.}\PY{n}{xlabel}\PY{p}{(}\PY{l+s+s2}{\PYZdq{}}\PY{l+s+s2}{theta\PYZus{}1}\PY{l+s+s2}{\PYZdq{}}\PY{p}{)}
          \PY{n}{plt}\PY{o}{.}\PY{n}{ylabel}\PY{p}{(}\PY{l+s+s2}{\PYZdq{}}\PY{l+s+s2}{dtheta\PYZus{}1}\PY{l+s+s2}{\PYZdq{}}\PY{p}{)}
          
          \PY{c+c1}{\PYZsh{} Don\PYZsq{}t plot a quiver arrow at *every* point, but instead}
          \PY{c+c1}{\PYZsh{} every ds points}
          \PY{c+c1}{\PYZsh{} (lower = more quiver arrows)}
          \PY{n}{ds} \PY{o}{=} \PY{l+m+mi}{3}
          \PY{n}{plt}\PY{o}{.}\PY{n}{quiver}\PY{p}{(}\PY{n}{Xplot}\PY{p}{[}\PY{p}{:}\PY{p}{:}\PY{n}{ds}\PY{p}{,} \PY{p}{:}\PY{p}{:}\PY{n}{ds}\PY{p}{]}\PY{p}{,} 
                     \PY{n}{Yplot}\PY{p}{[}\PY{p}{:}\PY{p}{:}\PY{n}{ds}\PY{p}{,} \PY{p}{:}\PY{p}{:}\PY{n}{ds}\PY{p}{]}\PY{p}{,} 
                     \PY{n}{f\PYZus{}samples}\PY{p}{[}\PY{p}{:}\PY{p}{:}\PY{n}{ds}\PY{p}{,} \PY{p}{:}\PY{p}{:}\PY{n}{ds}\PY{p}{,} 
                     \PY{n}{theta2d\PYZus{}plotting\PYZus{}slice}\PY{p}{,} \PY{l+m+mi}{0}\PY{p}{]}\PY{p}{,} \PY{n}{f\PYZus{}samples}\PY{p}{[}\PY{p}{:}\PY{p}{:}\PY{n}{ds}\PY{p}{,} \PY{p}{:}\PY{p}{:}\PY{n}{ds}\PY{p}{,} \PY{n}{theta2d\PYZus{}plotting\PYZus{}slice}\PY{p}{,} \PY{l+m+mi}{2}\PY{p}{]}\PY{p}{)}\PY{p}{;}
\end{Verbatim}


    \begin{center}
    \adjustimage{max size={0.9\linewidth}{0.9\paperheight}}{output_40_0.png}
    \end{center}
    { \hspace*{\fill} \\}
    
    \subsection{3.6 LQR Region of Attraction, Episode 2 (6 points, 3/6
autograded)}\label{lqr-region-of-attraction-episode-2-6-points-36-autograded}

Now you should have access to (many samples from) your candidate \(V\)
and \(\dot{V}\). One approach you could use from here is to estimate the
region of attraction from these samples. Let's do that -\/- \textbf{use
Theorem 10.5 from the textbook to estimate the largest region of
attraction that you can for the LQR controller you derived above -\/-
that is, find the biggest value of \(\rho\) such that at every point
\(x_i\) where \(V(x_i) \leq \rho\), \(\dot{V}(x_i) \prec 0\).}

(Hint: it'll be easiest to do this by finding counterexample -\/- e.g.,
the point \(x_i\) with the smallest \(V(x_i)\) where
\(\dot{V}(x_i) \geq 0\).)

    \begin{Verbatim}[commandchars=\\\{\}]
{\color{incolor}In [{\color{incolor}336}]:} \PY{k}{def} \PY{n+nf}{estimate\PYZus{}rho}\PY{p}{(}\PY{n}{V}\PY{p}{,} \PY{n}{Vdot}\PY{p}{)}\PY{p}{:}
              \PY{l+s+sd}{\PYZsq{}\PYZsq{}\PYZsq{}}
          \PY{l+s+sd}{    Code submission for 3.6}
          \PY{l+s+sd}{    Fill in this function to use the samples of V and Vdot}
          \PY{l+s+sd}{    (Each array has dimension [n\PYZus{}bins, n\PYZus{}bins, n\PYZus{}bins\PYZus{}theta2d])}
          \PY{l+s+sd}{    to compute a maximal rho indicating the region of attraction}
          \PY{l+s+sd}{    of the fixed point at the upright.}
          \PY{l+s+sd}{    \PYZsq{}\PYZsq{}\PYZsq{}}
              \PY{n}{a} \PY{o}{=} \PY{n}{np}\PY{o}{.}\PY{n}{min}\PY{p}{(}\PY{n}{V}\PY{p}{[}\PY{n}{np}\PY{o}{.}\PY{n}{logical\PYZus{}and}\PY{p}{(}\PY{n}{Vdot} \PY{o}{\PYZgt{}}\PY{o}{=} \PY{l+m+mi}{0}\PY{p}{,} \PY{n}{V} \PY{o}{!=} \PY{l+m+mi}{0}\PY{p}{)}\PY{p}{]}\PY{p}{)}
              \PY{n}{x} \PY{o}{=} \PY{n}{np}\PY{o}{.}\PY{n}{max}\PY{p}{(}\PY{n}{V}\PY{p}{[}\PY{n}{np}\PY{o}{.}\PY{n}{logical\PYZus{}and}\PY{p}{(}\PY{n}{Vdot} \PY{o}{\PYZlt{}} \PY{l+m+mi}{0}\PY{p}{,} \PY{n}{V} \PY{o}{\PYZlt{}}\PY{o}{=} \PY{n}{a}\PY{p}{)}\PY{p}{]}\PY{p}{)}
              \PY{k}{return} \PY{n}{x}
          
          \PY{n}{rho} \PY{o}{=} \PY{n}{estimate\PYZus{}rho}\PY{p}{(}\PY{n}{V\PYZus{}samples}\PY{p}{,} \PY{n}{Vdot\PYZus{}samples}\PY{p}{)}
          \PY{k}{print} \PY{l+s+s2}{\PYZdq{}}\PY{l+s+s2}{Region of attraction estimated at V(x) \PYZlt{}= }\PY{l+s+s2}{\PYZdq{}}\PY{p}{,} \PY{n}{rho}
          
          \PY{c+c1}{\PYZsh{} Plot Vdot again, but overlay the region of attraction \PYZhy{}\PYZhy{} which,}
          \PY{c+c1}{\PYZsh{} for quadratic V, is an ellipse.}
          \PY{n}{fig} \PY{o}{=} \PY{n}{plt}\PY{o}{.}\PY{n}{figure}\PY{p}{(}\PY{p}{)}
          \PY{n}{ax} \PY{o}{=} \PY{n}{fig}\PY{o}{.}\PY{n}{add\PYZus{}subplot}\PY{p}{(}\PY{l+m+mi}{1}\PY{p}{,} \PY{l+m+mi}{1}\PY{p}{,} \PY{l+m+mi}{1}\PY{p}{)}
          \PY{n}{fig}\PY{o}{.}\PY{n}{set\PYZus{}size\PYZus{}inches}\PY{p}{(}\PY{l+m+mi}{12}\PY{p}{,}\PY{l+m+mi}{6}\PY{p}{)}
          \PY{n}{plt}\PY{o}{.}\PY{n}{pcolormesh}\PY{p}{(}\PY{n}{Xplot}\PY{p}{,} \PY{n}{Yplot}\PY{p}{,} \PY{n}{Vdot\PYZus{}viz}\PY{p}{,} \PY{n}{vmin}\PY{o}{=}\PY{l+m+mi}{0}\PY{p}{,} \PY{n}{vmax}\PY{o}{=}\PY{l+m+mf}{1.0}\PY{p}{)}
          
          \PY{c+c1}{\PYZsh{} The part of S we care about is the 2x2 submatrix from the 1st and 3rd rows}
          \PY{c+c1}{\PYZsh{} and columns.}
          \PY{n}{S\PYZus{}sub} \PY{o}{=} \PY{n}{np}\PY{o}{.}\PY{n}{reshape}\PY{p}{(}\PY{n}{S}\PY{p}{[}\PY{p}{[}\PY{l+m+mi}{0}\PY{p}{,} \PY{l+m+mi}{2}\PY{p}{,} \PY{l+m+mi}{0}\PY{p}{,} \PY{l+m+mi}{2}\PY{p}{]}\PY{p}{,} \PY{p}{[}\PY{l+m+mi}{0}\PY{p}{,} \PY{l+m+mi}{0}\PY{p}{,} \PY{l+m+mi}{2}\PY{p}{,} \PY{l+m+mi}{2}\PY{p}{]}\PY{p}{]}\PY{p}{,} \PY{p}{(}\PY{l+m+mi}{2}\PY{p}{,} \PY{l+m+mi}{2}\PY{p}{)}\PY{p}{)}
          \PY{c+c1}{\PYZsh{} Extract its eigenvalues and eigenvectors, which tell us}
          \PY{c+c1}{\PYZsh{} the axes of the ellipse}
          \PY{n}{ellipseInfo} \PY{o}{=} \PY{n}{np}\PY{o}{.}\PY{n}{linalg}\PY{o}{.}\PY{n}{eig}\PY{p}{(}\PY{n}{S\PYZus{}sub}\PY{p}{)}
          \PY{c+c1}{\PYZsh{} Eigenvalues are 1/r\PYZca{}2, Eigenvectors are axis directions}
          \PY{n}{axis\PYZus{}1} \PY{o}{=} \PY{n}{ellipseInfo}\PY{p}{[}\PY{l+m+mi}{1}\PY{p}{]}\PY{p}{[}\PY{l+m+mi}{0}\PY{p}{,} \PY{p}{:}\PY{p}{]}
          \PY{k}{if} \PY{n}{ellipseInfo}\PY{p}{[}\PY{l+m+mi}{0}\PY{p}{]}\PY{p}{[}\PY{l+m+mi}{0}\PY{p}{]} \PY{o}{\PYZgt{}} \PY{l+m+mi}{0} \PY{o+ow}{and} \PY{n}{ellipseInfo}\PY{p}{[}\PY{l+m+mi}{0}\PY{p}{]}\PY{p}{[}\PY{l+m+mi}{1}\PY{p}{]} \PY{o}{\PYZgt{}} \PY{l+m+mi}{0}\PY{p}{:}
              \PY{n}{r1} \PY{o}{=} \PY{n}{math}\PY{o}{.}\PY{n}{sqrt}\PY{p}{(}\PY{n}{rho}\PY{p}{)}\PY{o}{/}\PY{n}{math}\PY{o}{.}\PY{n}{sqrt}\PY{p}{(}\PY{n}{ellipseInfo}\PY{p}{[}\PY{l+m+mi}{0}\PY{p}{]}\PY{p}{[}\PY{l+m+mi}{0}\PY{p}{]}\PY{p}{)}
              \PY{n}{axis\PYZus{}2} \PY{o}{=} \PY{n}{ellipseInfo}\PY{p}{[}\PY{l+m+mi}{1}\PY{p}{]}\PY{p}{[}\PY{l+m+mi}{1}\PY{p}{,} \PY{p}{:}\PY{p}{]}
              \PY{n}{r2} \PY{o}{=} \PY{n}{math}\PY{o}{.}\PY{n}{sqrt}\PY{p}{(}\PY{n}{rho}\PY{p}{)}\PY{o}{/}\PY{n}{math}\PY{o}{.}\PY{n}{sqrt}\PY{p}{(}\PY{n}{ellipseInfo}\PY{p}{[}\PY{l+m+mi}{0}\PY{p}{]}\PY{p}{[}\PY{l+m+mi}{1}\PY{p}{]}\PY{p}{)}
              \PY{n}{angle} \PY{o}{=} \PY{n}{math}\PY{o}{.}\PY{n}{atan2}\PY{p}{(}\PY{o}{\PYZhy{}}\PY{n}{axis\PYZus{}1}\PY{p}{[}\PY{l+m+mi}{1}\PY{p}{]}\PY{p}{,} \PY{n}{axis\PYZus{}1}\PY{p}{[}\PY{l+m+mi}{0}\PY{p}{]}\PY{p}{)}
              \PY{k+kn}{from} \PY{n+nn}{matplotlib.patches} \PY{k+kn}{import} \PY{n}{Ellipse}
              \PY{n}{ax}\PY{o}{.}\PY{n}{add\PYZus{}patch}\PY{p}{(}\PY{n}{Ellipse}\PY{p}{(}\PY{p}{(}\PY{n}{xf}\PY{p}{[}\PY{l+m+mi}{0}\PY{p}{]}\PY{p}{,} \PY{n}{xf}\PY{p}{[}\PY{l+m+mi}{2}\PY{p}{]}\PY{p}{)}\PY{p}{,} 
                                   \PY{l+m+mi}{2}\PY{o}{*}\PY{n}{r1}\PY{p}{,} \PY{l+m+mi}{2}\PY{o}{*}\PY{n}{r2}\PY{p}{,} 
                                   \PY{n}{angle}\PY{o}{=}\PY{n}{angle}\PY{o}{*}\PY{l+m+mf}{180.}\PY{o}{/}\PY{n}{math}\PY{o}{.}\PY{n}{pi}\PY{p}{,} 
                                   \PY{n}{linewidth}\PY{o}{=}\PY{l+m+mi}{2}\PY{p}{,} \PY{n}{fill}\PY{o}{=}\PY{n+nb+bp}{False}\PY{p}{,} \PY{n}{zorder}\PY{o}{=}\PY{l+m+mi}{2}\PY{p}{)}\PY{p}{)}\PY{p}{;}
              
              \PY{c+c1}{\PYZsh{} Report an interesting number that is easy to compute}
              \PY{c+c1}{\PYZsh{} from the ellipse info}
              \PY{k}{print} \PY{l+s+s2}{\PYZdq{}}\PY{l+s+s2}{Area of your region of attraction: }\PY{l+s+s2}{\PYZdq{}}\PY{p}{,} \PY{n}{math}\PY{o}{.}\PY{n}{pi} \PY{o}{*} \PY{n}{r1} \PY{o}{*} \PY{n}{r2}
          \PY{k}{else}\PY{p}{:}
              \PY{k}{print} \PY{l+s+s2}{\PYZdq{}}\PY{l+s+s2}{S\PYZus{}sub had nonpositive eigenvalues. That shouldn}\PY{l+s+s2}{\PYZsq{}}\PY{l+s+s2}{t happen.}\PY{l+s+s2}{\PYZdq{}}
              
          \PY{n}{plt}\PY{o}{.}\PY{n}{title}\PY{p}{(}\PY{l+s+s2}{\PYZdq{}}\PY{l+s+s2}{Vdot(x) overlaid with estimated ROA at dtheta\PYZus{}2 = }\PY{l+s+si}{\PYZpc{}f}\PY{l+s+s2}{\PYZdq{}} \PY{o}{\PYZpc{}} \PY{n}{z}\PY{p}{[}\PY{n}{theta2d\PYZus{}plotting\PYZus{}slice}\PY{p}{]}\PY{p}{)}
          \PY{n}{plt}\PY{o}{.}\PY{n}{xlabel}\PY{p}{(}\PY{l+s+s2}{\PYZdq{}}\PY{l+s+s2}{theta\PYZus{}1}\PY{l+s+s2}{\PYZdq{}}\PY{p}{)}
          \PY{n}{plt}\PY{o}{.}\PY{n}{ylabel}\PY{p}{(}\PY{l+s+s2}{\PYZdq{}}\PY{l+s+s2}{dtheta\PYZus{}1}\PY{l+s+s2}{\PYZdq{}}\PY{p}{)}\PY{p}{;}
\end{Verbatim}


    \begin{Verbatim}[commandchars=\\\{\}]
Region of attraction estimated at V(x) <=  1.42696950421
Area of your region of attraction:  0.0703475988692

    \end{Verbatim}

    \begin{center}
    \adjustimage{max size={0.9\linewidth}{0.9\paperheight}}{output_42_1.png}
    \end{center}
    { \hspace*{\fill} \\}
    
    \subsection{3.7 LQR Region of Attraction, Intermission (3
points)}\label{lqr-region-of-attraction-intermission-3-points}

\begin{enumerate}
\def\labelenumi{\arabic{enumi})}
\item
  Is this sufficient proof that LQR \emph{would} work starting from
  \(x_0\) \emph{inside} region of attraction (ignoring sampling errors)?
\item
  What about the opposite -\/- is this sufficient proof that LQR
  \emph{would not} work starting from \(x_0\) \emph{outside} of this
  region of attraction?
\item
  If you answer to (2) is No, speculate what strategies we could use to
  find larger regions of attraction. (For example, could changing V
  increase the largest ROA we could guarantee? How could we formulate
  search over V?)
\end{enumerate}

    \textbf{Short answer explanation for 3.7}

\begin{enumerate}
\def\labelenumi{\arabic{enumi})}
\item
  If you are in the region of attraction (assuming that you correctly
  sampled/etc), your Lyaponov analysis guarantees that LQR will work.
\item
  No, your region of attraction does not guarantee that outside of it,
  it will not work (you could've picked a too-conservative V).
\item
  Yes, we could change V. We could for instance imagine doing a search
  over all SOS functions (ensuring positivity), persuant to relevant
  conditions in our problem (such as the negative gradient), that
  maximizes the region that has a negative gradient.
\end{enumerate}

    \subsection{3.8 Combined Swing-up and Stabilization (8 points, 4/8
autograded)}\label{combined-swing-up-and-stabilization-8-points-48-autograded}

Finally, we're ready for the main event! \textbf{Write a controller that
will accomplish the task of swinging up the inertial pendulum to its
upright fixed point from any initial condition. Your controller should
use a custom swingup controller to get close to the unstable fixed
point, and then switch to LQR to stay there. Use your estimated region
of attraction to decide when it's safe to switch to LQR. A framework for
setting up this hybrid controller is provided below for your
convenience.}

To accomplish swing-up, apply energy shaping to \(\theta\) via
non-collocated feedback linearization. (Energy shaping for \(\theta\)
should follow from the simple pendulum, so you just need figure out how
to use your non-collocated input \(\tau\) to directly apply torque on
\(\theta\).)

While we'll look at your code, we'll also rely on the autograder to test
your system from a variety of initial conditions. (You can try the
autograder with the cell at the bottom of the notebook.) Besides the
random initial conditions we'll test, we'll also check: - \$x =
\left[ 0, 0, 0, 0 \right] \$ - \$x = \left[ \pi, 0, 0, 0 \right] \$ -
\$x = \left[ 3\pi, 0, 0, 0 \right] \$ - \$x =
\left[ 0, -100, 0, 0 \right] \$ - \$x = \left[ 0, 0, 0, 20 \right] \$

Make sure you can handle these corner cases!

    \begin{Verbatim}[commandchars=\\\{\}]
{\color{incolor}In [{\color{incolor}395}]:} \PY{c+c1}{\PYZsh{} The swingup controller should accept a state x,}
          \PY{c+c1}{\PYZsh{} and return a control input u (a 1x1 numpy array)}
          \PY{c+c1}{\PYZsh{} that respects the plant\PYZsq{}s input limits.}
          
          \PY{k}{def} \PY{n+nf}{swingup\PYZus{}controller}\PY{p}{(}\PY{n}{x}\PY{p}{)}\PY{p}{:}
              \PY{c+c1}{\PYZsh{} Here\PYZsq{}s some useful things...}
              \PY{n}{q} \PY{o}{=} \PY{n}{x}\PY{p}{[}\PY{l+m+mi}{0}\PY{p}{:}\PY{l+m+mi}{2}\PY{p}{]}
              \PY{n}{qd} \PY{o}{=} \PY{n}{x}\PY{p}{[}\PY{l+m+mi}{2}\PY{p}{:}\PY{l+m+mi}{4}\PY{p}{]}
              \PY{p}{(}\PY{n}{M}\PY{p}{,} \PY{n}{C}\PY{p}{,} \PY{n}{tauG}\PY{p}{,} \PY{n}{B}\PY{p}{)} \PY{o}{=} \PY{n}{pendulum\PYZus{}plant}\PY{o}{.}\PY{n}{GetManipulatorDynamics}\PY{p}{(}\PY{n}{q}\PY{p}{,} \PY{n}{qd}\PY{p}{)}
              
              \PY{n}{wheelKinetic} \PY{o}{=} \PY{n}{M}\PY{p}{[}\PY{l+m+mi}{1}\PY{p}{,}\PY{l+m+mi}{1}\PY{p}{]} \PY{o}{*} \PY{n}{qd}\PY{p}{[}\PY{l+m+mi}{1}\PY{p}{]} \PY{o}{*} \PY{n}{qd}\PY{p}{[}\PY{l+m+mi}{1}\PY{p}{]}
              \PY{n}{armKinetic} \PY{o}{=} \PY{n}{M}\PY{p}{[}\PY{l+m+mi}{0}\PY{p}{,}\PY{l+m+mi}{0}\PY{p}{]} \PY{o}{*} \PY{n}{qd}\PY{p}{[}\PY{l+m+mi}{0}\PY{p}{]} \PY{o}{*}  \PY{n}{qd}\PY{p}{[}\PY{l+m+mi}{0}\PY{p}{]}
              
              \PY{n}{potential} \PY{o}{=} \PY{o}{\PYZhy{}}\PY{p}{(}\PY{n}{pendulum\PYZus{}plant}\PY{o}{.}\PY{n}{m1}\PY{o}{*}\PY{n}{pendulum\PYZus{}plant}\PY{o}{.}\PY{n}{l1} \PY{o}{+} \PY{n}{pendulum\PYZus{}plant}\PY{o}{.}\PY{n}{m2}\PY{o}{*}\PY{n}{pendulum\PYZus{}plant}\PY{o}{.}\PY{n}{l2}\PY{p}{)}\PY{o}{*}\PY{n}{pendulum\PYZus{}plant}\PY{o}{.}\PY{n}{g}\PY{o}{*}\PY{n}{np}\PY{o}{.}\PY{n}{cos}\PY{p}{(}\PY{n}{qd}\PY{p}{[}\PY{l+m+mi}{0}\PY{p}{]}\PY{p}{)}
              
              \PY{c+c1}{\PYZsh{} If close to the top, attempt LQR if wheel slow enough, or slow down wheel}
              \PY{n}{eps} \PY{o}{=} \PY{l+m+mf}{0.22}
              \PY{k}{if} \PY{n+nb}{abs}\PY{p}{(}\PY{p}{(}\PY{n}{q}\PY{p}{[}\PY{l+m+mi}{0}\PY{p}{]} \PY{o}{\PYZpc{}} \PY{l+m+mi}{2}\PY{o}{*}\PY{n}{np}\PY{o}{.}\PY{n}{pi}\PY{p}{)} \PY{o}{\PYZhy{}} \PY{n}{np}\PY{o}{.}\PY{n}{pi}\PY{p}{)} \PY{o}{\PYZlt{}} \PY{n}{eps} \PY{o+ow}{or} \PY{n+nb}{abs}\PY{p}{(}\PY{p}{(}\PY{n}{q}\PY{p}{[}\PY{l+m+mi}{0}\PY{p}{]} \PY{o}{\PYZpc{}} \PY{l+m+mi}{2}\PY{o}{*}\PY{n}{np}\PY{o}{.}\PY{n}{pi}\PY{p}{)} \PY{o}{+} \PY{n}{np}\PY{o}{.}\PY{n}{pi}\PY{p}{)} \PY{o}{\PYZlt{}} \PY{n}{eps}\PY{p}{:}
                  \PY{k}{if} \PY{n}{wheelKinetic} \PY{o}{\PYZgt{}} \PY{l+m+mi}{80}\PY{p}{:}
                      \PY{k}{return} \PY{n}{np}\PY{o}{.}\PY{n}{clip}\PY{p}{(}\PY{n}{np}\PY{o}{.}\PY{n}{array}\PY{p}{(}\PY{p}{[}\PY{o}{\PYZhy{}}\PY{n}{qd}\PY{p}{[}\PY{l+m+mi}{1}\PY{p}{]} \PY{o}{*} \PY{l+m+mi}{100}\PY{p}{]}\PY{p}{)}\PY{p}{,} \PY{o}{\PYZhy{}}\PY{n}{pendulum\PYZus{}plant}\PY{o}{.}\PY{n}{input\PYZus{}max}\PY{p}{,} \PY{n}{pendulum\PYZus{}plant}\PY{o}{.}\PY{n}{input\PYZus{}max}\PY{p}{)}
                  \PY{k}{else}\PY{p}{:}
                      \PY{k}{return} \PY{n}{lqr\PYZus{}controller}\PY{p}{(}\PY{n}{x}\PY{p}{)}
              
              \PY{c+c1}{\PYZsh{} Decrease wheel kinetic energy to be reasonable}
              \PY{k}{if} \PY{n}{wheelKinetic} \PY{o}{\PYZgt{}} \PY{l+m+mi}{30}\PY{p}{:}
                  \PY{k}{return} \PY{n}{np}\PY{o}{.}\PY{n}{clip}\PY{p}{(}\PY{n}{np}\PY{o}{.}\PY{n}{array}\PY{p}{(}\PY{p}{[}\PY{o}{\PYZhy{}}\PY{n}{qd}\PY{p}{[}\PY{l+m+mi}{1}\PY{p}{]} \PY{o}{*} \PY{l+m+mi}{100}\PY{p}{]}\PY{p}{)}\PY{p}{,} \PY{o}{\PYZhy{}}\PY{n}{pendulum\PYZus{}plant}\PY{o}{.}\PY{n}{input\PYZus{}max}\PY{p}{,} \PY{n}{pendulum\PYZus{}plant}\PY{o}{.}\PY{n}{input\PYZus{}max}\PY{p}{)}
              \PY{c+c1}{\PYZsh{} Decrease energy of the swing to just make us reach the top}
              \PY{k}{elif} \PY{n}{armKinetic} \PY{o}{+} \PY{n}{potential} \PY{o}{\PYZgt{}} \PY{l+m+mi}{175}\PY{p}{:}
                  \PY{k}{return} \PY{n}{np}\PY{o}{.}\PY{n}{clip}\PY{p}{(}\PY{n}{np}\PY{o}{.}\PY{n}{array}\PY{p}{(}\PY{p}{[}\PY{n}{qd}\PY{p}{[}\PY{l+m+mi}{0}\PY{p}{]} \PY{o}{*} \PY{l+m+mi}{100} \PY{o}{+} \PY{n}{tauG}\PY{p}{[}\PY{l+m+mi}{0}\PY{p}{]}\PY{p}{]}\PY{p}{)}\PY{p}{,} \PY{o}{\PYZhy{}}\PY{n}{pendulum\PYZus{}plant}\PY{o}{.}\PY{n}{input\PYZus{}max}\PY{p}{,} \PY{n}{pendulum\PYZus{}plant}\PY{o}{.}\PY{n}{input\PYZus{}max}\PY{p}{)}    
              
              \PY{c+c1}{\PYZsh{}Force a swing to get energy}
              \PY{k}{if} \PY{p}{(}\PY{n}{qd}\PY{p}{[}\PY{l+m+mi}{0}\PY{p}{]} \PY{o}{\PYZgt{}}\PY{o}{=}\PY{l+m+mi}{0}\PY{p}{)} \PY{p}{:}
                  \PY{k}{return} \PY{n}{np}\PY{o}{.}\PY{n}{array}\PY{p}{(}\PY{p}{[}\PY{o}{\PYZhy{}}\PY{n}{pendulum\PYZus{}plant}\PY{o}{.}\PY{n}{input\PYZus{}max}\PY{p}{]}\PY{p}{)}
              \PY{k}{else}\PY{p}{:}
                  \PY{k}{return} \PY{n}{np}\PY{o}{.}\PY{n}{array}\PY{p}{(}\PY{p}{[}\PY{o}{+}\PY{n}{pendulum\PYZus{}plant}\PY{o}{.}\PY{n}{input\PYZus{}max}\PY{p}{]}\PY{p}{)}
          
          \PY{k}{def} \PY{n+nf}{combined\PYZus{}controller}\PY{p}{(}\PY{n}{x}\PY{p}{)}\PY{p}{:}
              \PY{l+s+sd}{\PYZsq{}\PYZsq{}\PYZsq{} }
          \PY{l+s+sd}{    Code response for problem 3.8:}
          \PY{l+s+sd}{    Fill in this combined controller to dispatch to either the}
          \PY{l+s+sd}{    swingup or lqr controller by checking the state against your}
          \PY{l+s+sd}{    calculated region of attraction.}
          \PY{l+s+sd}{    \PYZsq{}\PYZsq{}\PYZsq{}}
              \PY{k}{if} \PY{n}{calcV}\PY{p}{(}\PY{n}{x}\PY{p}{)} \PY{o}{\PYZlt{}} \PY{n}{rho}\PY{p}{:}
                  \PY{k}{return} \PY{n}{lqr\PYZus{}controller}\PY{p}{(}\PY{n}{x}\PY{p}{)}
              \PY{k}{else}\PY{p}{:}
                  \PY{k}{return} \PY{n}{swingup\PYZus{}controller}\PY{p}{(}\PY{n}{x}\PY{p}{)}
          
          \PY{c+c1}{\PYZsh{} Simulate!}
          \PY{c+c1}{\PYZsh{}x0 = [2.0, 0.0, 30, 0.]}
          \PY{n}{x0} \PY{o}{=} \PY{p}{[}\PY{l+m+mi}{0}\PY{p}{,}\PY{l+m+mi}{0}\PY{p}{,} \PY{l+m+mi}{0}\PY{p}{,} \PY{l+m+mi}{0}\PY{p}{]}
          \PY{c+c1}{\PYZsh{}x0 = [2.0, 10.0, \PYZhy{}0.5, 0]}
          \PY{c+c1}{\PYZsh{}x0 = [2.0, 0.0, \PYZhy{}0.5, \PYZhy{}100.]}
          \PY{c+c1}{\PYZsh{}x0 = [2.0, 0.0, \PYZhy{}0.5, 100.]}
          \PY{c+c1}{\PYZsh{}x0 = [0.0, 0.0, 0.0, 0.]}
          \PY{c+c1}{\PYZsh{}x0 = [np.pi, 0, 0, 0]}
          \PY{c+c1}{\PYZsh{}x0 = [3*np.pi, 0, 0, 0]}
          \PY{c+c1}{\PYZsh{}x0 = [0, \PYZhy{}100, 0, 0]}
          \PY{c+c1}{\PYZsh{}x0 = [0, 0, 0, 20]}
          
          \PY{n}{duration} \PY{o}{=} \PY{l+m+mi}{50}
          \PY{n}{input\PYZus{}log}\PY{p}{,} \PY{n}{state\PYZus{}log} \PY{o}{=} \PY{n}{RunSimulation}\PY{p}{(}\PY{n}{pendulum\PYZus{}plant}\PY{p}{,}
                                  \PY{n}{combined\PYZus{}controller}\PY{p}{,}
                                  \PY{n}{x0} \PY{o}{=} \PY{n}{x0}\PY{p}{,}
                                  \PY{n}{duration} \PY{o}{=} \PY{n}{duration}\PY{p}{)}
          
          \PY{c+c1}{\PYZsh{} Plot traces of the results}
          \PY{k+kn}{import} \PY{n+nn}{matplotlib.pyplot} \PY{k+kn}{as} \PY{n+nn}{plt}
          \PY{n}{fig} \PY{o}{=} \PY{n}{plt}\PY{o}{.}\PY{n}{figure}\PY{p}{(}\PY{p}{)}
          \PY{n}{fig}\PY{o}{.}\PY{n}{set\PYZus{}size\PYZus{}inches}\PY{p}{(}\PY{l+m+mi}{12}\PY{p}{,}\PY{l+m+mi}{12}\PY{p}{)}
          \PY{k}{for} \PY{n}{i} \PY{o+ow}{in} \PY{n+nb}{range}\PY{p}{(}\PY{l+m+mi}{4}\PY{p}{)}\PY{p}{:}
              \PY{n}{plt}\PY{o}{.}\PY{n}{subplot}\PY{p}{(}\PY{l+m+mi}{5}\PY{p}{,} \PY{l+m+mi}{1}\PY{p}{,} \PY{n}{i}\PY{o}{+}\PY{l+m+mi}{1}\PY{p}{)}
              \PY{n}{plt}\PY{o}{.}\PY{n}{plot}\PY{p}{(}\PY{n}{state\PYZus{}log}\PY{o}{.}\PY{n}{sample\PYZus{}times}\PY{p}{(}\PY{p}{)}\PY{p}{,} \PY{n}{state\PYZus{}log}\PY{o}{.}\PY{n}{data}\PY{p}{(}\PY{p}{)}\PY{p}{[}\PY{n}{i}\PY{p}{,} \PY{p}{:}\PY{p}{]}\PY{p}{)}
              \PY{n}{plt}\PY{o}{.}\PY{n}{grid}\PY{p}{(}\PY{n+nb+bp}{True}\PY{p}{)}
              \PY{n}{plt}\PY{o}{.}\PY{n}{ylabel}\PY{p}{(}\PY{l+s+s2}{\PYZdq{}}\PY{l+s+s2}{x[}\PY{l+s+si}{\PYZpc{}d}\PY{l+s+s2}{]}\PY{l+s+s2}{\PYZdq{}} \PY{o}{\PYZpc{}} \PY{n}{i}\PY{p}{)}
          \PY{n}{plt}\PY{o}{.}\PY{n}{subplot}\PY{p}{(}\PY{l+m+mi}{5}\PY{p}{,} \PY{l+m+mi}{1}\PY{p}{,} \PY{l+m+mi}{5}\PY{p}{)}
          \PY{n}{plt}\PY{o}{.}\PY{n}{plot}\PY{p}{(}\PY{n}{input\PYZus{}log}\PY{o}{.}\PY{n}{sample\PYZus{}times}\PY{p}{(}\PY{p}{)}\PY{p}{,} \PY{n}{input\PYZus{}log}\PY{o}{.}\PY{n}{data}\PY{p}{(}\PY{p}{)}\PY{p}{[}\PY{l+m+mi}{0}\PY{p}{,} \PY{p}{:}\PY{p}{]}\PY{p}{)}
          \PY{n}{plt}\PY{o}{.}\PY{n}{ylabel}\PY{p}{(}\PY{l+s+s2}{\PYZdq{}}\PY{l+s+s2}{u[0]}\PY{l+s+s2}{\PYZdq{}}\PY{p}{)}
          \PY{n}{plt}\PY{o}{.}\PY{n}{xlabel}\PY{p}{(}\PY{l+s+s2}{\PYZdq{}}\PY{l+s+s2}{t}\PY{l+s+s2}{\PYZdq{}}\PY{p}{)}
          \PY{n}{plt}\PY{o}{.}\PY{n}{grid}\PY{p}{(}\PY{n+nb+bp}{True}\PY{p}{)}
\end{Verbatim}


    \begin{center}
    \adjustimage{max size={0.9\linewidth}{0.9\paperheight}}{output_46_0.png}
    \end{center}
    { \hspace*{\fill} \\}
    
    \begin{Verbatim}[commandchars=\\\{\}]
{\color{incolor}In [{\color{incolor}377}]:} \PY{n}{np}\PY{o}{.}\PY{n}{linalg}\PY{o}{.}\PY{n}{pinv}\PY{p}{(}\PY{n}{np}\PY{o}{.}\PY{n}{array}\PY{p}{(}\PY{p}{[}\PY{p}{[}\PY{l+m+mi}{0}\PY{p}{,}\PY{l+m+mi}{1}\PY{p}{]}\PY{p}{]}\PY{p}{)}\PY{p}{)}
\end{Verbatim}


\begin{Verbatim}[commandchars=\\\{\}]
{\color{outcolor}Out[{\color{outcolor}377}]:} array([[ 0.],
                 [ 1.]])
\end{Verbatim}
            
    \begin{Verbatim}[commandchars=\\\{\}]
{\color{incolor}In [{\color{incolor}363}]:} \PY{k+kn}{from} \PY{n+nn}{IPython.display} \PY{k+kn}{import} \PY{n}{HTML}
          \PY{k+kn}{from} \PY{n+nn}{inertial\PYZus{}wheel\PYZus{}pendulum\PYZus{}visualizer} \PY{k+kn}{import} \PY{o}{*}
          
          \PY{n}{viz} \PY{o}{=} \PY{n}{InertialWheelPendulumVisualizer}\PY{p}{(}\PY{n}{pendulum\PYZus{}plant}\PY{p}{)}
          \PY{n}{ani} \PY{o}{=} \PY{n}{viz}\PY{o}{.}\PY{n}{animate}\PY{p}{(}\PY{n}{input\PYZus{}log}\PY{p}{,} \PY{n}{state\PYZus{}log}\PY{p}{,} \PY{l+m+mi}{30}\PY{p}{,} \PY{n}{repeat}\PY{o}{=}\PY{n+nb+bp}{True}\PY{p}{)}
          \PY{n}{plt}\PY{o}{.}\PY{n}{close}\PY{p}{(}\PY{n}{viz}\PY{o}{.}\PY{n}{fig}\PY{p}{)}
          \PY{n}{HTML}\PY{p}{(}\PY{n}{ani}\PY{o}{.}\PY{n}{to\PYZus{}html5\PYZus{}video}\PY{p}{(}\PY{p}{)}\PY{p}{)}
\end{Verbatim}


\begin{Verbatim}[commandchars=\\\{\}]
{\color{outcolor}Out[{\color{outcolor}363}]:} <IPython.core.display.HTML object>
\end{Verbatim}
            
    \subsection{Test your own
implementations}\label{test-your-own-implementations}

Running the cell below will run your implemented functions against unit
tests.

Don't change the cell below, or the test\_set\_3.py file. We will grade
your implementations against the original files.

Make sure to SAVE your notebook before running tests. (File
-\/-\textgreater{} Save and Checkpoint, or use the hotkey which should
be ctrl+s on linux, cmd+s on osx, etc)

\textbf{Warning: these take a little while to run, so run them
sparingly!} (They test your code by testing the inertial pendulum from
lots of initial conditions...) You can speed it up a little by
decreasing the time it takes to run every part of the notebook (e.g.
sample V less frequently in 3.5, simulate plants for a shorter
duration.)

    \begin{Verbatim}[commandchars=\\\{\}]
{\color{incolor}In [{\color{incolor}396}]:} \PY{k+kn}{import} \PY{n+nn}{os}
          \PY{c+c1}{\PYZsh{} Run the test in a subprocess, to make sure it doesn\PYZsq{}t open any plots...}
          \PY{n}{os}\PY{o}{.}\PY{n}{popen}\PY{p}{(}\PY{l+s+s2}{\PYZdq{}}\PY{l+s+s2}{python test\PYZus{}set\PYZus{}3.py ./ test\PYZus{}results.json}\PY{l+s+s2}{\PYZdq{}}\PY{p}{)}
          
          \PY{c+c1}{\PYZsh{} Print the results json for review}
          \PY{k+kn}{import} \PY{n+nn}{test\PYZus{}set\PYZus{}3}
          \PY{k}{print} \PY{n}{test\PYZus{}set\PYZus{}3}\PY{o}{.}\PY{n}{pretty\PYZus{}format\PYZus{}json\PYZus{}results}\PY{p}{(}\PY{l+s+s2}{\PYZdq{}}\PY{l+s+s2}{test\PYZus{}results.json}\PY{l+s+s2}{\PYZdq{}}\PY{p}{)}
\end{Verbatim}


    \begin{Verbatim}[commandchars=\\\{\}]
Test Problem 2\_1: Finding Q: 4.00/4.00.

Test Problem 3\_1: Linearization, A matrix: 2.00/2.00.
  * [[ 50.   0.]  [  0.   0.]]

Test Problem 3\_1: Linearization, B matrix: 2.00/2.00.
  * [[ 50.   0.]  [  0.   0.]]

Test Problem 3.2: Controllability: 2.00/2.00.
  * [[ 50.   0.]  [  0.   0.]] [[-50.   0.]  [  0.   0.]]

Test Problem 3.3: LQR: 4.00/4.00.
  * [[ 50.   0.]  [  0.   0.]]

Test Problem 3.4: LQR ROA, Prologue: 2.00/2.00.

Test Problem 3.5: LQR ROA, Evaluating F, V, and Vdot: 2.00/2.00.

Test Problem 3.6: LQR ROA, Numerical estimate of the ROA: 3.00/3.00.

Test Problem 3.8: Combined Swing-Up and Stabilization: 4.00/4.00.

TOTAL SCORE (automated tests only): 25.00/25.00


    \end{Verbatim}

    \subsection{Feedback Survey}\label{feedback-survey}

We'd really appreciate your feedback on this set, and on the class so
far! Anonymous survey
\href{https://goo.gl/forms/gSqeSnMdY8WHeCiC2}{available here}.


    % Add a bibliography block to the postdoc
    
    
    
    \end{document}
